Проект реализован посредством языка программирования Java. Java – объектно-ориентированный язык программирования, разработанный компанией Sun Microsystems (в последующем приобретённой компанией Oracle). Приложения Java обычно транслируются в специальный байт-код, поэтому они могут работать на любой виртуальной Java-машине вне зависимости от компьютерной архитектуры. Дата официального выпуска – 23 мая 1995 года.
Программы на Java транслируются в байт-код, выполняемый виртуальной машиной Java (JVM) – программой, обрабатывающей байтовый код и передающей инструкции оборудованию как интерпретатор.
Достоинством подобного способа выполнения программ является полная независимость байт-кода от операционной системы и оборудования, что позволяет выполнять Java-приложения на любом устройстве, для которого существует соответствующая виртуальная машина. Другой важной особенностью технологии Java является гибкая система безопасности, в рамках которой исполнение программы полностью контролируется виртуальной машиной. Любые операции, которые превышают установленные полномочия программы (например, попытка несанкционированного доступа к данным или соединения с другим компьютером), вызывают немедленное прерывание.
Часто к недостаткам концепции виртуальной машины относят снижение производительности. Ряд усовершенствований несколько увеличил скорость выполнения программ на Java:
1 Применение технологии трансляции байт-кода в машинный код непосредственно во время работы программы (JIT-технология) с возможностью сохранения версий класса в машинном коде,
2 Широкое использование платформенно-ориентированного кода (native-код) в стандартных библиотеках [20].
3 Аппаратные средства, обеспечивающие ускоренную обработку байт-кода (например, технология Jazelle, поддерживаемая некоторыми процессорами фирмы ARM).
По данным сайта “shootout.alioth.debian.org”, для семи разных задач время выполнения на Java составляет в среднем в полтора-два раза больше, чем для C/C++, в некоторых случаях Java быстрее, а в отдельных случаях в 7 раз медленнее. С другой стороны, для большинства из них потребление памяти Java-машиной было в 10–30 раз больше, чем программой на C/C++. Также примечательно исследование, проведённое компанией Google, согласно которому отмечается существенно более низкая производительность и большее потребление памяти в тестовых примерах на Java в сравнении с аналогичными программами на C++.
Идеи, заложенные в концепцию и различные реализации среды виртуальной машины Java, вдохновили множество энтузиастов на расширение перечня языков, которые могли бы быть использованы для создания программ, исполняемых на виртуальной машине. Эти идеи нашли также выражение в спецификации общеязыковой инфраструктуры CLI, заложенной в основу платформы. Внутри Java существуют несколько основных семейств технологий:

\begin{enumerate}
	\item Java SE – Java Standard Edition, основное издание Java, содержит компиляторы, API, Java Runtime Environment; подходит для создания пользовательских приложений, в первую очередь – для настольных систем.
	\item Java EE – Java Enterprise Edition, представляет собой набор спецификаций для создания программного обеспечения уровня предприятия.
	\item Java ME – Java Micro Edition, создана для использования в устройствах, ограниченных по вычислительной мощности, например, в мобильных телефонах, КПК, встроенных системах;
	\item JavaFX – технология, являющаяся следующим шагом в эволюции Java как Rich Client Platform; предназначена для создания графических интерфейсов корпоративных приложений и бизнеса.
	\item Java Card – технология предоставляет безопасную среду для приложений, работающих на смарт-картах и других устройствах с очень ограниченным объёмом памяти и возможностями обработки. .NET компанией Microsoft.
\end{enumerate}

Следующие успешные проекты реализованы с привлечением Java (J2EE) технологий: «RuneScape», «Amazon», «eBay», «LinkedIn», «Yahoo!».
Следующие компании в основном фокусируются на Java (J2EE) технологиях: SAP, IBM, Oracle. В частности, СУБД Oracle Database включает JVM как свою составную часть, обеспечивающую возможность непосредственного программирования СУБД на языке Java, включая, например, хранимые процедуры.
Программы, написанные на Java, имеют репутацию более медленных и занимающих больше оперативной памяти, чем написанные на языке C. Тем не менее, скорость выполнения программ, написанных на языке Java, была существенно улучшена с выпуском в 1997–1998 годах так называемого JIT-компилятора в версии 1.1 в дополнение к другим особенностям языка для поддержки лучшего анализа кода (такие, как внутренние классы, класс StringBuffer, упрощенные логические вычисления и т. д.). Кроме того, была произведена оптимизация виртуальной машины Java – с 2000 года для этого используется виртуальная машина HotSpot. По состоянию на февраль 2012 года, код Java 7 приблизительно в 1.8 раза медленнее кода, написанного на языке Си.
Некоторые платформы предлагают аппаратную поддержку выполнения для Java. К примеру, микроконтроллеры, выполняющие код Java на аппаратном обеспечении вместо программной JVM, а также основанные на ARM процессоры, которые поддерживают выполнение байткода Java через опцию Jazelle.

Основные возможности:

\begin{enumerate}
	\item Автоматическое управление памятью;
	\item Расширенные возможности обработки исключительных ситуаций;
	\item Богатый набор средств фильтрации ввода-вывода;
	\item Набор стандартных коллекций: массив, список, стек и т. п.;
	\item Наличие простых средств создания сетевых приложений (в том числе с использованием протокола RMI);
	\item Наличие классов, позволяющих выполнять HTTP-запросы и обрабатывать ответы;
	\item Встроенные в язык средства создания многопоточных приложений, которые потом были портированы на многие языки (например, python);
	\item Унифицированный доступ к базам данных на уровне отдельных SQL-запросов – на основе JDBC, SQLJ;
	\item Поддержка обобщений (начиная с версии 1.5);
	\item Поддержка лямбд, замыканий, встроенные возможности функционального программирования (с версии 1.8);
	\item Наличие вариантов реализации многопоточных программ
\end{enumerate}

Разработчику на Java доступно множество готовых (или библиотечных) классов и методов, полезных для использования в собственных программах. Наличие библиотечных решений позволяет изящно решать множество задач. Рассматриваемый компонент позволит преобразовать вычисления в программный код.

\textit{Spring Framework} (или коротко \textit{Spring}) — универсальный фреймворк с открытым исходным кодом для Java-платформы. 

Первая версия была написана Родом Джонсоном, который впервые опубликовал её вместе с изданием своей книги «\textit{Expert One-on-One Java EE Design and Development}»[3] (Wrox Press, октябрь 2002 года).

Фреймворк был впервые выпущен под лицензией Apache 2.0 license в июне 2003 года. Первая стабильная версия 1.0 была выпущена в марте 2004. Spring 2.0 был выпущен в октябре 2006, Spring 2.5 — в ноябре 2007, Spring 3.0 в декабре 2009, и Spring 3.1 в декабре 2011. Текущая версия — 5.1.2.

Несмотря на то, что Spring не обеспечивал какую-либо конкретную модель программирования, он стал широко распространённым в Java-сообществе главным образом как альтернатива и замена модели Enterprise JavaBeans. Spring предоставляет большую свободу Java-разработчикам в проектировании. Кроме того, он предоставляет хорошо документированные и лёгкие в использовании средства решения проблем, возникающих при создании приложений корпоративного масштаба.

Между тем, особенности ядра Spring применимы в любом Java-приложении, и существует множество расширений и усовершенствований для построения веб-приложений на Java Enterprise платформе. По этим причинам Spring приобрёл большую популярность и признаётся разработчиками как стратегически важный фреймворк.

Spring обеспечивает решения многих задач, с которыми сталкиваются Java-разработчики и организации, которые хотят создать информационную систему, основанную на платформе Java. Из-за широкой функциональности трудно определить наиболее значимые структурные элементы, из которых он состоит. Spring не всецело связан с платформой Java Enterprise, несмотря на его масштабную интеграцию с ней, что является важной причиной его популярности.

Spring, вероятно, наиболее известен как источник расширений (features), нужных для эффективной разработки сложных бизнес-приложений вне тяжеловесных программных моделей, которые исторически были доминирующими в промышленности. Ещё одно его достоинство в том, что он ввел ранее неиспользуемые функциональные возможности в сегодняшние господствующие методы разработки, даже вне платформы Java.

Этот фреймворк предлагает последовательную модель и делает её применимой к большинству типов приложений, которые уже созданы на основе платформы Java. Считается, что Spring реализует модель разработки, основанную на лучших стандартах индустрии, и делает её доступной во многих областях Java.

Spring может быть рассмотрен как коллекция меньших фреймворков или фреймворков во фреймворке. Большинство этих фреймворков может работать независимо друг от друга, однако они обеспечивают большую функциональность при совместном их использовании. Эти фреймворки делятся на структурные элементы типовых комплексных приложений:

\begin{enumerate}
	\item Inversion of Control-контейнер: конфигурирование компонентов приложений и управление жизненным циклом Java-объектов.
	\item Фреймворк аспектно-ориентированного программирования: работает с функциональностью, которая не может быть реализована возможностями объектно-ориентированного программирования на Java без потерь.
	\item Фреймворк доступа к данным: работает с системами управления реляционными базами данных на Java-платформе, используя JDBC- и ORM-средства и обеспечивая решения задач, которые повторяются в большом числе Java-based environments.
	\item Фреймворк управления транзакциями: координация различных API управления транзакциями и инструментарий настраиваемого управления транзакциями для объектов Java.
	\item Фреймворк MVC: каркас, основанный на HTTP и сервлетах, предоставляющий множество возможностей для расширения и настройки (customization).
	\item Фреймворк удалённого доступа: конфигурируемая передача Java-объектов через сеть в стиле RPC, поддерживающая RMI, CORBA, HTTP-based протоколы, включая web-сервисы (SOAP).
	\item Фреймворк аутентификации и авторизации: конфигурируемый инструментарий процессов аутентификации и авторизации, поддерживающий много популярных и ставших индустриальными стандартами протоколов, инструментов, практик через дочерний проект Spring Security (ранее известный как Acegi).
	\item Фреймворк удалённого управления: конфигурируемое представление и управление Java-объектами для локальной или удалённой конфигурации с помощью JMX.
	\item Фреймворк работы с сообщениями: конфигурируемая регистрация объектов-слушателей сообщений для прозрачной обработки сообщений из очереди сообщений с помощью JMS, улучшенная отправка сообщений по стандарту JMS API.
	\item Тестирование: каркас, поддерживающий классы для написания модульных и интеграционных тестов.
\end{enumerate}

Центральной частью Spring является контейнер Inversion of Control, который предоставляет средства конфигурирования и управления объектами Java с помощью рефлексии. Контейнер отвечает за управление жизненным циклом объекта: создание объектов, вызов методов инициализации и конфигурирование объектов путём связывания их между собой.

Объекты, создаваемые контейнером, также называются управляемыми объектами (beans). Обычно конфигурирование контейнера осуществляется путём загрузки XML-файлов, содержащих определение bean’ов и предоставляющих информацию, необходимую для создания bean’ов.

Spring имеет собственную MVC-платформу веб-приложений, которая не была первоначально запланирована. Разработчики Spring решили написать её как реакцию на то, что они восприняли как неудачность конструкции (тогда) популярного Apache Struts, а также других доступных веб-фреймворков. В частности, по их мнению, было недостаточным разделение между слоями представления и обработки запросов, а также между слоем обработки запросов и моделью.

Класс DispatcherServlet является основным контроллером фрэймворка и отвечает за делегирование управления различным интерфейсам, на всех этапах выполнения HTTP-запроса. Об этих интерфейсах следует сказать более подробно.

Как и Struts, Spring MVC является фреймворком, ориентированным на запросы. В нем определены стратегические интерфейсы для всех функций современной запросно-ориентированной системы. Цель каждого интерфейса — быть простым и ясным, чтобы пользователям было легко его заново имплементировать, если они того пожелают. MVC прокладывает путь к более чистому front-end-коду. Все интерфейсы тесно связаны с Servlet API. Эта связь рассматривается некоторыми как неспособность разработчиков Spring предложить для веб-приложений абстракцию более высокого уровня. Однако эта связь оставляет особенности Servlet API доступными для разработчиков, облегчая все же работу с ним. 

Docker — программное обеспечение для автоматизации развёртывания и управления приложениями в средах с поддержкой контейнеризации. Позволяет «упаковать» приложение со всем его окружением и зависимостями в контейнер, который может быть перенесён на любую Linux-систему с поддержкой cgroups в ядре, а также предоставляет среду по управлению контейнерами. Изначально использовал возможности LXC, с 2015 года применял собственную библиотеку, абстрагирующую виртуализационные возможности ядра Linux — libcontainer. С появлением ​Open Container Initiative начался переход от монолитной к модульной архитектуре.

Программное обеспечение функционирует в среде Linux с ядром, поддерживающим cgroups и изоляцию пространств имён (namespaces); существуют сборки только для платформ x86-64 и ARM[17]. Начиная с версии 1.6 возможно использование в ОС Windows.

Для экономии дискового пространства проект использует файловую систему Aufs с поддержкой технологии каскадно-объединённого монтирования: контейнеры используют образ базовой операционной системы, а изменения записываются в отдельную область. Также поддерживается размещение контейнеров в файловой системе Btrfs с включённым режимом копирования при записи.

В состав программных средств входит демон — сервер контейнеров, клиентские средства, позволяющие из интерфейса командной строки управлять образами и контейнерами, а также API, позволяющий в стиле REST управлять контейнерами программно.

Демон обеспечивает полную изоляцию запускаемых на узле контейнеров на уровне файловой системы (у каждого контейнера собственная корневая файловая система), на уровне процессов (процессы имеют доступ только к собственной файловой системе контейнера, а ресурсы разделены средствами libcontainer), на уровне сети (каждый контейнер имеет доступ только к привязанному к нему сетевому пространству имён и соответствующим виртуальным сетевым интерфейсам).

Набор клиентских средств позволяет запускать процессы в новых контейнерах (docker run), останавливать и запускать контейнеры (docker stop и docker start), приостанавливать и возобновлять процессы в контейнерах (docker pause и docker unpause). Серия команд позволяет осуществлять мониторинг запущенных процессов (docker ps по аналогии с ps в Unix-системах, docker top по аналогии с top и другие). Новые образы возможно создавать из специального сценарного файла (docker build, файл сценария носит название Dockerfile), возможно записать все изменения, сделанные в контейнере, в новый образ (docker commit). Все команды могут работать как с docker-демоном локальной системы, так и с любым сервером Docker, доступным по сети. Кроме того, в интерфейсе командной строки встроены возможности по взаимодействию с публичным репозиторием Docker Hub, в котором размещены предварительно собранные образы контейнеров, например, команда docker search позволяет осуществить поиск образов среди размещённых в нём, образы можно скачивать в локальную систему (docker pull), возможно также отправить локально собранные образы в Docker Hub (docker push).

Также Docker имеет пакетный менеджер Docker Compose, позволяющий описывать и запускать многоконтейнерные приложения. Конфигурационные файлы Compose описываются на языке YAML.

Amazon Web Services (AWS) – наиболее распространенная в мире облачная платформа с самыми широкими возможностями, которая предоставляет 165 полнофункциональных сервисов для центров обработки данных по всей планете. Миллионы клиентов, в том числе стартапы, ставшие лидерами по скорости роста, крупнейшие корпорации и передовые правительственные учреждения, доверяют AWS в вопросах размещения инфраструктуры, повышения гибкости и снижения затрат.

AWS предоставляет сервисы для широкого спектра приложений, включая вычислительные сервисы, сервисы хранилищ, баз данных, сетевых конфигураций, аналитики, машинного обучения и искусственного интеллекта, Интернета вещей (IoT), обеспечения безопасности, сервисы для разработки и развертывания приложений, а также управления ими.

AWS обеспечивает не только самый большой спектр сервисов, но и самые широкие функциональные возможности в их рамках. Например, Amazon EC2 предлагает больше типов и размеров вычислительных инстансов, чем любой другой поставщик, в том числе самые мощные инстансы с графическими процессорами для рабочих нагрузок, связанных с машинными обучением. AWS также обеспечивает вдвое больше сервисов баз данных, чем ближайшие конкуренты, и предлагает одиннадцать реляционных и нереляционных баз данных. К тому же, AWS обеспечивает больше всего способов запуска контейнеров: с помощью Amazon Elastic Container Service (ECS), Amazon Elastic Container Service for Kubernetes (EKS) и AWS Fargate.

Широкий выбор сервисов и разнообразные функциональные возможности обеспечивают более простую, быструю и экономичную миграцию существующих приложений и предоставляют почти безграничные возможности для разработки.

% Java book
% Spring book
% Docker book
% AWS Book https://aws.amazon.com/ru/what-is-aws/#most-functionality