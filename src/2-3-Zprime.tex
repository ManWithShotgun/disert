В физических программаx экспериментов на  современных  дронных (\textit{LHC}) и планируемых на  электрон-позитронных (\textit{ILC, CLIC}) коллайдерах вопросу поиск  <<новой>> физики, выходящей за  рамки Стандартной модели (СМ), традиционно уделяется большое внимание. К числу подобных теоретических построений, являющихся обобщением СМ, относятся модели с расширенным колибровочным сектором, такие как лево-правосимметричные модели (\textit{LR}), альтернативные лево-правосимметричные модели (\textit{ALR}), $E_6$-модели
и др.~\cite{Bobovnikov:2016}. Их исследование (теоретическое и экспериментальное) представляет значительный интерес. Эти модели являются одними из простейших расширений СМ, характеризующихся элементарной структурой хиггсовского сектора. Общим для данных моделей является то, что они предсказывают новые физические объекты и явления на масштабе энергий $O$ (1 ТэВ), связанные, например, с наличием тяжелых нейтральных ($Z^\prime$) калибровочных бозонов, обусловленных дополнительными калибровочными симметриями $U(1)^\prime$.

Достижение порога рождения $Z^\prime$-бозона явилось бы прямым доказательством проявления «новой» физики. Однако в данном случае интервал поиска масс $Z^\prime$ ограничен максимальной энергией коллайдера, на котором проводятся эксперименты. Значительно более широкий интервал масс можно исследовать с помощью пропагаторных эффектов. В этом случае ведется поиск отклонений различных наблюдаемых от соответствующих предсказаний СМ. Если экспериментальные данные при достигнутом уровне точности согласуются с СМ, т. е. отклонений от предсказаний СМ нет, то эту экспериментальную информацию можно использовать для получения ограничений на динамические параметры и массы $Z^\prime$-бозонов.

Потенциальные возможности $e^+$$e^-$-коллайдеров для прямого рождения новых калибровочных бозонов гораздо скромнее по сравнению с адронными машинами из-за более низких энергий пучков. Кроме того, современные ограничения на массы $Z^\prime$-бозонов для большинства моделей превосходят планируемую энергию электрон-позитронного коллайдера \textit{ILC}, $\sqrt{s}<< M_{Z^\prime}$. Тем не менее основным достоинством этих машин является возможность проведения экспериментов по измерению наблюдаемых величин с высокой степенью точности и получения однозначной информации о косвенных (виртуальных) эффектах новых $Z^\prime$-бозонов, а также эффектах бозонного $Z$-$Z^\prime$-смешивания. Последние, в моделях с расширенным калибровочным сектором, зависят от структуры хиггсовского сектора модели. Тем самым экспериментальное исследование процессов рождения пар $W^±$-бозонов может не только пролить свет на возможное существование <<новой>> физики, но и дать косвенные указания на хиггсовскую природу, а также установить структуру модели.

На основе данных, полученных из низкоэнергетических экспериментов по нейтральным токам, результатов на $e^+e^-$-коллайдерах \textit{LEP} и \textit{SLC}~\cite{Bobovnikov:2016}, а также недавно выполненных экспериментов по поиску прямого адронного рождения $Z^\prime$-бозонов в процессе Дрелла-Яна:
\begin{equation} \label{eq:drell}
	pp \rightarrow Z^\prime \rightarrow l^+l^- + X
\end{equation}
где $l=e,\mu$) на коллайдере \textit{LНC} при энергии $\sqrt{s}$ = 7 и 8 ТэВ с интегральной светимостью соответственно $L_int$ = 5 и 20 фб${}^{-1}$~\cite{Bobovnikov:2016} можно заключить, что для большинства расширенных калибровочных моделей граничные значения для масс дополнительных $Z^\prime$- бозонов находятся в интервале $\sim$ 2,5-3,0 ТэВ (в зависимости от модели), а современный масштаб ограничений на угол смешивания составляет $\mathcal{O}(\varphi )$ ~ ${10}^{-2}$--${10}^{-3}$ рад. При этом наиболее точная информация об угле смешивания была получена преимущественно из экспериментов на электрон-позитронных коллайдерах \textit{LEP1}~\cite{schael:2006} и \textit{SLC} по измерению резонансных наблюдаемых физических величин при энергии начальных состояний, равной массе стандартного $Z$-бозона, $\sqrt{s}$ = $M_Z$, в процессах:
\begin{equation} \label{eq:drell2}
	e^+e^- \rightarrow f\bar{f}
\end{equation}
где конечными фермионными состояниями $f$ были заряженные лептоны и кварки~\cite{andreev-pankov:2012}. Высокая точность, достигнутая в экспериментах на коллайдерах \textit{LEP1} и \textit{SLC}, объясняется прежде всего возможностью набора большого объема данных в резонансной области энергии.

Кроме того, эта информация дополнялась данными, полученными на коллайдере тэватрон, по точному измерению массы $M_W$, на основе которых определялся параметр бозонного $Z$−$Z^\prime$-смешивания с использованием соотношения между массами нейтральных и заряженных калибровочных бозонов, $M_Z$ = $M_W$/$(\sqrt{p_0}\cos\theta_W)$, имеющего место в расширенных моделях. Очевидно также, что эти данные будут дополнены новой информацией, которая в ближайшем будущем будет получена в экспериментах на коллайдере \textit{LНС} при энергии 13 и 14 ТэВ. Вместе с тем из этих данных нельзя сделать однозначный вывод о природе «новой» физики, который мог бы вызвать отклонение наблюдаемых величин от их поведения, предсказываемого СМ. Дело в том, что параметр $p$, который содержится в выражениях для векторных и аксиально-векторных констант связи фермионов с учетом петлевых поправок, зависит, в частности, от структуры хиггсовского сектора модели, которая изначально неизвестна. Кроме того, новые тяжелые фермионы и скалярные частицы, предсказываемые моделями с расширенным калибровочным сектором, могут давать вклад в параметр $p$ на петлевом уровне. Все эти неопределенности приводят к появлению систематических (теоретических) погрешностей, которые могут быть весьма существенными при измерении параметра $p$ и, в конечном счете, могут повлиять на точность определения параметра $Z$−$Z^\prime$-смешивания.

Процессы парного рождения заряженных $W^±$-бозонов в адронных столкновениях на \textit{LНС}:
\begin{equation} \label{eq:drell3}
	pp \rightarrow W^+W^- + X
\end{equation}

Процессы электрон-позитронной аннигиляции на \textit{LЕР2} и в большей степени на \textit{ILС}:
\begin{equation} \label{eq:drell4}
	e^+e^- \rightarrow W^+W^-
\end{equation}

Являются весьма эффективным инструментом поиска эффектов $Z$−$Z^\prime$-смешивания при высоких энергиях и, таким образом, играют роль основного поставщика информации об угле $Z$−$Z^\prime$-смешивания~\cite{Bobovnikov:2016}. С теоретической точки зрения процессы парного рождения заряженных калибровочных бозонов в адронных и электронпозитронных столкновениях интересны тем, что их сечения пропорциональны углу $Z$−$Z^\prime$-смешивания, который, как отмечалось выше, в расширенных калибровочных моделях зависит от структуры сектора Хиггса~\cite{sirunyan:2017}.

Прямой поиск тяжелых резонансов в процессе $p\bar{p} \rightarrow W^+W^- + X$ осуществлялся экспериментальными группами \textit{СDF} и \textit{D0} на коллайдере тэватрон. Коллаборация \textit{D0} исследовала возможность рождения резонанса в канале его дибозонного распада, используя чисто лептонные $lvl^\prime v^\prime$ и полулептонные $vjj$ моды. Здесь $l=e,\mu$; $jj$ — две адронные струи. Коллаборация \textit{СDF} также осуществляла поиск тяжелых резонансов в канале их распада в пару заряженных калибровочных бозонов $W^+W^−$ с последующим распадом в полулептонные $evjj$ конечные состояния. Обе коллаборации установили ограничения на массы тяжелых резонансов, таких как новые нейтральные $Z^\prime$- и заряженные калибровочные $W^±$-бозоны, гравитоны Рэндалл-Сандрума. Кроме того, в настоящее время поиск тяжелых резонансов на \textit{LHC} в \textit{WW}-канале интенсивно ведется коллаборациями \textit{ATLAS} и \textit{CMS}. В частности, уже получена экспериментальная информация о процессе в лептонном канале $lvl^\prime v^\prime$ при энергии коллайдера 7 ТэВ и интегральной светимости 4,7 фб${}^{−1}$~\cite{2part-pankov}.

Из анализа экспериментальных данных по измерению процесса электрон-позитронной аннигиляции на коллайдере \textit{LEP2} были впервые получены прямые ограничения на угол $Z$−$Z^\prime$-смешивания. Точность измерения угла смешивания оказалась не очень высокой, $\left |\phi \right |$~5—10 \%, так как сам коллайдер работал в интервале энергий, незначительно превышающем порог реакции, $\sqrt{s} >> 2M_W$. Как было установлено ранее, чувствительность процесса электрон-позитронной аннигиляции к эффектам «новой» физики значительно усиливается при высоких энергиях, $\sqrt{s} >> 2M_W$, где важную роль играет механизм калибровочного сокращения. Дело в том, что вклад $Z^\prime$-бозона в сечение процесса нарушает механизм калибровочного сокращения, играющий важную роль в СМ~\cite{andreev-ee:2012}. Действие механизма калибровочного сокращения состоит в том, что он обеспечивает «правильное» поведение сечения процесса электрон-позитронной аннигиляции с ростом энергии, которое не нарушает унитарный предел, несмотря на быстро растущие с энергией отдельные вклады в сечение. Вместе с тем эффекты, индуцированные появлением дополнительного калибровочного бозона, нарушают механизм калибровочного сокращения в энергетическом интервале $2M_W << \sqrt{s} << M_{Z^\prime}$, что проявляется в виде «разбалансировки» отдельных вкладов в сечение и, как следствие, в возникновении существенно иной по сравнению со СМ энергетической зависимостью сечений. Этим обусловлено действие так называемого механизма усиления эффектов «новой» физики в процессе электрон-позитронной аннигиляции. Именно в силу этого обстоятельства линейный коллайдер \textit{ILC} является одним из основных инструментариев для поиска эффектов «новой» физики при исследовании процесса электрон-позитронной аннигиляции.

Следует отметить также, что коллаборация \textit{CDF} на коллайдере тэватрон одной из первых получила прямые ограничения на угол $Z$−$Z^\prime$-смешивания из обработки данных по измерению процесса адронного рождения $W^+W^−$-бозонов. И вновь относительно небольшая энергия установки и низкая светимость не позволили улучшить ограничения, полученные на коллайдере \textit{LEP2}, а лишь повторить их~\cite{ada-wwz:2013}.

Возможности коллайдера \textit{LНС} по обнаружению эффектов $Z$−$Z^\prime$-смешивания в процессе рождения пар заряженных калибровочных $W^±$-бозонов с их последующим распадом по чисто лептонному каналу $lvl^\prime v^\prime$. Несмотря на очевидное достоинство данного канала, связанное с подавленностью фона, особенно при больших инвариантных массах $W^±$-бозонов, у него имеется заметный недостаток, связанный с присутствием в конечных фермионных состояниях двух нейтрино, что не позволяет восстановить распределение по инвариантной массе бозонных пар из экспериментальных данных. В то же время распад пары $W^±$-бозонов по полулептонному каналу $lvjj$ свободен от указанного недостатка. В процессе $pp \rightarrow Z^\prime \rightarrow WW + X \rightarrow lvjj + X$ существует возможность реконструировать распределение по инвариантной массе $W^+W^-$- пары и тем самым исследовать резонансную структуру $Z^\prime$-бозона. Еще одним достоинством настоящего полулептонного процесса является то, что он имеет сечение, существенно превосходящее сечение чисто лептонного канала. Вместе с тем полулептонный канал, в отличие от лептонного канала $lvl^\prime v^\prime$, имеет большой КХД-фон, вызванный рождением $W_{jj}$-, а также $Z_{jj}$-состояний~\cite{ada-lvlv:2013}. В последнем случае предполагается, что $Z$-бозон распадается по лептонному каналу, а в процессе детектирования лептонов один из них теряется. Кроме перечисленных выше КХД фоновых процессов имеется еще один, который играет важную роль в оценке всей фоновой составляющей. Это процесс рождения пар $t\bar{t}$-кварков. Однако большой КХД-фон может быть редуцирован путем наложения кинематических ограничений на поперечные импульсы заряженных лептонов и адронных струй в резонансном сигнале рождения $Z^\prime$-бозонов~\cite{Bobovnikov:2016}.

% ---

Дифференциальное сечение рождения ${Z}^{\prime}$ в процесс $pp \rightarrow W^+W^- + X$ из начальных кварк-антикварковых состояний может быть выражено как:
\begin{equation} \label{dsigma}
\frac{d\sigma^{Z^\prime}}{dM\,dy\,dz}
= K \frac{2 M}{s}
\sum_q [f_{q|P_1}(\xi_1)f_{\bar q|P_2}(\xi_2) + f_{\bar
	q|P_1}(\xi_1)f_{q|P_2}(\xi_2)]\, \frac{d\hat \sigma_{q \bar
		q}^{Z^\prime}}{dz}.
\end{equation}

Здесь, $s$ обозначает квадрат энергии в протон-протононном столкновении,
$z\equiv\cos\theta$, с улом $\theta$ для $W^-$-бозон-кварка $W^+W^-$ center-of-mass frame and $y$ скорость дибозона. Более того, функции $f_{q|P_1}(\xi_{1},M)$ и $f_{\bar
	q|P_2}(\xi_{2},M)$ являются распределения протонов для
протонного столкновения $P_1$ и $P_2$, соответственно, с моментом протона $\xi_{1,2}=(M/\sqrt
s)\exp(\pm y)$. Функция ${d\hat
	\sigma_{q \bar q}^{Z^\prime}}/{dz}$ является дифференциалом
поперечное сечение протона. В~(\ref{dsigma}), $K$ выражает фактор коэффициент вклада \textit{QCD} высоких порядков.
Для численного расчета, использовались партонные распределения \textit{CTEQ-6L1}~\cite{2part-pankov}. Оценки получены на уровне Борна,
следовательно, шкала факторизации $ \mu_{\rm F} $ входит исключительно через
функции распределения партонов, как пересечение партонного уровня
сечение в этом порядке не зависит от $ \mu_{\rm F} $. В отношении
масштаба зависимости партонных распределений, которые выбраны для
масштаба факторизации $ W ^ + W ^ - $ инвариантной масса для $ \mu _ {\ rm
	F} ^ 2 = M ^ 2 = \hat {s} $, с $ \hat {s} = \ xi_1 \, \ xi_2 \, s $ в партоном
подпроцессе. Полученные ограничения, представленные ниже, существенно не изменяются, когда
$ \mu _ {\rm F} $ варьируется от $ \mu _ {\rm F} / 2 $ до $ 2 \mu _ {\rm F}. $

Плотность вероятности рождения $ Z '$ бозона и
последующий распад в пару $ W ^ + W ^ - $, необходимый для оценки
ожидаемого числа событий $ Z '$, $ N ^ {Z ^ \prime} $, получен
из функции (\ref{dsigma}) путем интегрирования правой части по $ z $,
над скоростью $ W ^ \pm $ пары $ y $ и инвариантной массой $ M $
вокруг резонансного пика $ (M_R- \Delta M / 2, $ $ M_R + \Delta M / 2) $:
\begin{equation}
\sigma^{Z^\prime}{(pp\to W^+W^- + X)}  =\int_{M_{R}-\Delta
	M/2}^{M_{R}+\Delta M/2}d M \int_{-Y}^{Y}d y
\int_{-z_{cut}}^{z_{cut}}d
z\frac{d\sigma^{Z^\prime}}{d M\, d y\, d z}\;, \label{TotCr}
\end{equation}

Используя функцию~(\ref{TotCr}), число сигнальных событий для резонансного состояния $ Z '$ можно записать следующим образом:
\begin{align}
&N^{Z^\prime}= {\cal L}\cdot\varepsilon\cdot
\sigma^{Z^\prime}{(pp\to W^+W^- + X)}\nonumber \\ &\equiv {\cal
	L}\cdot\varepsilon\cdot A_{WW}\cdot \sigma(pp\to Z') \times {Br}(Z' \to W^+W^-).
\label{signal}
\end{align}

Здесь $ {\cal L} $ обозначает интегрированную светимость, а общая эффективность триггера кинематического и геометрического времени принятия, восстановления и выбора, $ A_ {WW} \times \varepsilon $, определяется как число сигнальных событий, проходящих через полный выбор событий, деленный на количество сгенерированных событий~\cite{2part-pankov}. Наконец, $ \sigma (pp \to Z ') \times {Br} (Z' \to W ^ + W ^ -) $ - это (теоретическое) отношение времени ветвления плотности сечения рождения, экстраполированное на полное фазовое пространство.

Дифференциальное сечение для процессов $ q \bar {q} \to Z '_ {\rm SSM} \to W ^ + W ^ - $, усредненных по кварковым цветам, можно записать в виде~\cite{2part-pankov}:

\begin{align}
&\frac{d\hat{\sigma}^{Z'}_{q \bar q}}{d \cos\theta}
= \frac{1}{3}\,\frac{\pi\alpha^2 \cot^2\theta_W}{16 }
\left(v_{f}^2 + a_{f}^2\right)\, \frac{\hat{s}}
{\left(\hat{s} - M_{Z'}^2\right)^2 + M_{Z'}^2\Gamma_{Z'}^2}  \nonumber \\
& \times  \xi^2\beta_W^3 \left(\frac{\hat{s}^2}{M_W^4}
\sin^2\theta +
4\frac{\hat{s}}{M_W^2}(4-\sin^2\theta)+12\sin^2\theta\right),
\label{xsection2}
\end{align}
где $v_f=(T_{3,f}-2Q_f\hskip 2pt s_W^2)/(2s_Wc_W)$;\\
$a_f=T_{3,f}/(2s_Wc_W)$;\\
$M_{Z'}$ -- масса;\\
$\Gamma_{Z'}$ -- плотность вероятности рождения $Z'$-бозона.

При расчете общей ширины $ \Gamma_ {Z '} $ включены следующие каналы: $ Z' \to f \bar f $, $ W ^ + W ^ - $ и $ ZH $ \cite{2part-pankov}, где $ H $ - бозон Хиггса СМ, а $ f $ - фермионы СМ ($ f = l, \nu, q $). Общая ширина $ \Gamma_ {Z '} $ бозона $ Z' $ может быть записана следующим образом:

\begin{equation}\label{gamma2}
\Gamma_{Z'} = \sum_f \Gamma_{Z'}^{ff} + \Gamma_{Z'}^{WW} +
\Gamma_{Z'}^{ZH}.
\end{equation}

Наличие двух последних каналов распада, которыми часто пренебрегают, связано с перемешиванием $ Z $ - $ Z '$. Однако для больших масс $ Z '$ существует усовершенствование, которое отменяет эффект из-за крошечного параметра смешивания ($ \xi $) для $ Z $ - $ Z' $~\cite{2part-pankov}. Необходимо отметить, что для всех интересующих научное сообщество значений $ M_ {Z '} $ для \textit{LHC} ширина бозона $ Z' _ {\rm SSM} $ значительно меньше, чем массовое разрешение $ \Delta M $.

Выражение для частичной ширины канала распада $ Z '\to W ^ + W ^ - $ можно записать в виде~\cite{2part-pankov}:

\begin{align}
&\Gamma_{Z'}^{WW}=\frac{\alpha}{48}\cot^2\theta_W\, M_{Z'}
\left(\frac{M_{Z'}}{M_W}\right)^4\left(1-4\,\frac{M_W^2}{M_{Z'}^2}\right)^{3/2} \nonumber \\
& \times \left[ 1+20 \left(\frac{M_W}{M_{Z'}}\right)^2 + 12
\left(\frac{M_W}{M_{Z'}}\right)^4\right]\xi^2 \label{GammaWW}
\end{align}

Доминирующий член во второй строке уравнения~(\ref{xsection2}) для $ M ^ 2 \gg M_W ^ 2 $ пропорционален $ (M / M_W) ^ 4 \sin ^ 2 \theta $ и соответствует рождению продольно поляризованных $ W $ и $ Z'\to W^+_LW^-_L $. Эта сильная зависимость от инвариантной массы приводит к очень крутому росту сечения с энергией и, следовательно, к значительному увеличению чувствительности сечения к перемешиванию $ Z $ - $ Z '$ при больших $ M $. В свою очередь, для фиксированного коэффициента смешивания $ \ xi $ и при больших $ M_ {Z '} $, где $ \Gamma_ {Z'} ^ {WW} $ доминирует над $ \sum_f \Gamma_ {Z '} ^ {ff } $ и $ \Gamma_ {Z '} ^ {ZH} $ полная ширина очень быстро увеличивается с массой $ M_ {Z'} $ из-за квинтической зависимости от $ Z '$ массы для $ W ^ + W ^ - $ распада, как показано в уравнении ~(\ref{GammaWW})
\cite{2part-pankov}. В этом случае вклад $ W ^ + W ^ - $ становится доминирующий, а $ {Br} (Z '\ to W ^ + W ^ -) \to 1 $, тогда как каналы фермионного распада все больше подавляются. Вместе с этим теорема об эквивалентности может предложить значение $ {Br} (Z'\to ZH) $, сравнимое с $ {Br} (Z' \to W ^ + W ^ -) $~\cite{2part-pankov}.