В рамках диссертационной работы проводилось имитационное моделирование процесса рождения ${Z}^{\prime}$-бозонов и распад ${W}^{+}{W}^{-}$ бозонов на Большом адронном коллайдере. В ходе моделирования были получены плотность вероятности рождения ${Z}^{\prime}$-бозонов для заданных входных параметров угла смешивания $\xi$ и массы бозона ${M}_{{Z}^{\prime}}$ с последующим распадом на пару ${W}^{+}{W}^{-}$ бозонов.

В ходе проведения вычислительных экспериментов было выявлено, что для получения дифференциального сечения (плотности вероятности) рождения ${Z}^{\prime}$-бозонов с точностью 90\% и выше необходимо было для каждой пары входных параметров угла смешивания $\xi$ и массы бозона ${M}_{{Z}^{\prime}}$ провести генерацию не меньше 100 000 событий столкновений протонов в процессе $pp \rightarrow W^+W^- + X$. 

Из-за присутствия статистической ошибки генератора даже при таком большом количестве генерируемых событий заданная точность не достигалась. Для уменьшения статистической ошибки проводились дополнительные итерации имитационного моделирования с постоянными значениями параметров угла смешивания $\xi$ и массы бозона ${M}_{{Z}^{\prime}}$, и разными значениями рандомизации для каждого цикла. Заданная точность достигалась при 10 000 циклов вычислительного эксперимента для одной точки на графике плотности вероятности рождения ${Z}^{\prime}$-бозонов.

В ходе моделирования было расчитанно 400 точек на плоскости параметров ($\sigma$ на ${M}_{{Z}^{\prime}}$) и смоделированно 