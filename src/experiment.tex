В рамках диссертационной работы проводилось имметационное моделирование процесса рождения ${Z}^{\prime}$-бозонов и распад ${W}^{+}{W}^{-}$ бозонов на Большом адронном коллайдере. В ходе моделирования были получены плотность вероятности прождения ${Z}^{\prime}$-бозонов для заданных входных параметров угла смешивания $\xi$ и массы бозона ${M}_{{Z}^{\prime}}$.

В ходе проведения иметационных моделированией было выявлено, что для полученя распределений плотности вероятности рождения ${Z}^{\prime}$-бозонов с точностью 90\% и выше необходимо было для каждой пары входных параметров угла смешивания $\xi$ и массы бозона ${M}_{{Z}^{\prime}}$ провести генерацию не меньше 100 000 событий стволкновений протонов в процессе $pp \rightarrow W^+W^- + X$. 

Из-за присутствия статистической ошибки генератора даже при таком большом количестве генерируемых событий заданая точность не достигалась. Для уменьшения статистической ошибки проводились дополнительные итерации иметационного моделирования с постоянными значениями параметров угла смешивания $\xi$ и массы бозона ${M}_{{Z}^{\prime}}$, и разными значениями рандомизации для каждого цикла. Заданная точность достигалась при 10 000 циклов ремоделирования одной точки на графике плотности вероятности рождения ${Z}^{\prime}$-бозонов.

как влияет кол-во данных на качество получаемых распределений