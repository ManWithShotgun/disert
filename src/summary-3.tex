%Данная глава посвящена имитационному моделированию процесса рождения ${Z}^{\prime}$-бозонов в протон-протнных столкновениях с учетом эффектов $Z$-${Z}^{\prime}$ смешивания в условиях эксперимента \textit{ATLAS} на Большом адронном коллайдере и разработке необходимых средств программного обеспечения.

В результате реализации предложенной математической модели процесса рождения ${Z}^{\prime}$-бозонов в протон-протнных столкновениях с последующим распадом на пару ${W}^{+}{W}^{-}$ бозонов и учета эффектов $Z$-${Z}^{\prime}$ смешивания были достигнуты следующие цели: 
\begin{enumerate}
	\item[--] реализована имитационная модель процесса рождения ${Z}^{\prime}$-бозонов в протон-протнных столкновениях с последующим распадом на пару ${W}^{+}{W}^{-}$ бозонов, отличающуюся от известных моделей учетом эффектов $Z$-${Z}^{\prime}$ смешивания;
	\item[--] реализованы необходимые программные средства для имитационного моделирования процесса
	рождения ${Z}^{\prime}$-бозонов в протон-протнных столкновениях с учетом эффектов $Z$-${Z}^{\prime}$ смешивания в условиях эксперимента \textit{ATLAS} на Большом адронном коллайдере, отличающиеся от существующих тем, что программный модуль собран в \textit{Docker} образ позволяющий быстро проводить имитационное моделирование без установки необходимых программных средств.
\end{enumerate}
