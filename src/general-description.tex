\chapter*{ОБЩАЯ ХАРАКТЕРИСТИКА РАБОТЫ}
\addcontentsline{toc}{chapter}{ОБЩАЯ ХАРАКТЕРИСТИКА РАБОТЫ} % in Content
\textbf{Связь работы с научными программами (проектами) и темами}\\

Диссертационная работа связана с тематикой НИР, выполняемых в рамках научно-исследовательского направления кафедры «Информационные технологии» Гомельского государственного технического университета им. П. О. Сухого.
Тема диссертации соответствует приоритетным направлениям
фундаментальных исследований в Республики Беларусь. Диссертационная
работа выполнялась в период с 2017 по 2019 годы в рамках отдельного
подзадания государственной программы научных исследований
«Конвергенция-2020», подзадание 2.1.05, номер гос. регистрации 20162284.
\\

\textbf{Цель и задачи исследования}\\

Целью работы является создание системы определяющий возможность рождения нового резонанса нейтрального спина 1 (${Z}^{\prime}$) 
из доступных данных групп \textit{ATLAS} для ${W}^{+}{W}^{-}$ распадов. В качестве результатов работы будут получены 
ограничения на соответствующие $Z$-${Z}^{\prime}$-коэффициенты смешивания и на массу $M_{Z^\prime}$.

Для достижения поставленной цели были поставлены следующие задачи:

\begin{enumerate}
	\item[--] изучить процесс рождения ${Z}^{\prime}$-бозонов и распад ${W}^{+}{W}^{-}$ бозонов на Большом адронном коллайдере;
	
	\item[--] разработать математическую модель процесса рождения ${Z}^{\prime}$-бозонов в протон-протонных столкновениях с учетом эффектов $Z$-${Z}^{\prime}$ смешивания;
	
	\item[--] выполнить разработку программного обеспечения для имитационного моделирования и оценки ограничений на соответствующие $Z$-${Z}^{\prime}$ параметры смешивания и на массу $M_{Z^\prime}$.
	
\end{enumerate}

Изучение появления электрослабых бозонов дает мощную проверку спонтанного нарушения 
калибровочной симметрии стандартной модели и может быть использовано для поиска новых явлений за пределами стандартной модели. 
Дополнительные нейтральные векторные бозоны ${Z}^{\prime}$, распадающиеся на заряженные пары калибровочных векторных бозонов ${W}^{+}{W}^{-}$, 
прогнозируются во многих сценариях новой физики, включая модели с расширенным калибровочным сектором.
\\

\textbf{Научная новизна}\\

Научная новизна работы заключается в том, что впервые получены ограничения на угл смешивания ${Z}^{\prime}$-бозонов в
процессе рождения ${W}^{+}$${W}^{-}$ пар в протон-протонных столкновениях для значений светимости 1000 фб${}^{−1}$ и 3000 фб${}^{−1}$ на Большом адронном коллайдере, а также создан программный модуль, позволяющий выполнять: имитационное моделирование рождения ${Z}^{\prime}$ в процессе ${W}^{+}{W}^{-}$
на Большом адронном коллайдере с учётом эффектов $Z$-${Z}^{\prime}$ смешивания.
\\

\textbf{Положения, выносимые на защиту}\\

Автором защищаются:
\begin{enumerate}
	\item[--] имитационная модель процесса рождения ${Z}^{\prime}$-бозонов в протон-протонных столкновениях с последующим распадом на пару ${W}^{+}{W}^{-}$ бозонов, отличающуюся от известных моделей учетом эффектов $Z$-${Z}^{\prime}$ смешивания;
	
	\item[--] программный модуль для имитационного моделирования процесса
	рождения ${Z}^{\prime}$-бозонов в протон-протонных столкновениях с учетом эффектов $Z$-${Z}^{\prime}$ смешивания в условиях эксперимента \textit{ATLAS} на Большом адронном коллайдере, отличающуюся от существующих тем, что программный модуль собран в \textit{Docker} образ позволяющий быстро проводить имитационное моделирование без установки необходимых программных средств;
	
	\item[--] оценки ограничений на углы смешивания ${Z}^{\prime}$-бозонов в
	процессе рождения ${W}^{+}$${W}^{-}$ пар в протон-протонных столкновениях
	в условиях экспериментов на Большом адронном коллайдере, рассчитаные для интегральной светимости 1000 фб${}^{−1}$ и 3000 фб${}^{−1}$, которые составили  ${10}^{-4}$ и $6\times{10}^{-5}$, соответственно, и существенно превышают существующие экспериментальные ограничения.
	
\end{enumerate}
\\

\textbf{Личный вклад соискателя}\\

Содержание диссертации целиком и полностью отображает личный вклад соискателя. Определение целей и задач исследования, обобщение полученных результатов проводилось совместно с научным руководителем К.С. Курочкой.

Диссертация прошла проверку на плагиат в системе обнаружения текстовых заимствований с результатом 80\% оригинальности.
\\

\textbf{Апробация результатов диссертации}\\

Результаты работы докладывались на студенческой международной научно-практической конференции V Международной конференции <<Инновации в современной науке>> (Киев, июнь 2019 г.).
\\

\textbf{Опубликованность результатов диссертации}\\

Результаты диссертационных исследований, связанных с измерением процесса рождения ${W}^{+}{W}^{-}$ пар в протон-протонных столкновениях и получены экспериментальные ограничения на угол смешивания ${Z}^{\prime}$-бозонов направлены в печать в рамках V Международной конференции <<Инновации в современной науке>> [1-A].
\\

\textbf{Структура и объем диссертации}\\

Диссертационная работа состоит из введения, четырёх глав, заключения и библиографического списка. Объем диссертации – 75 листов, включая 3 приложения и 46 иллюстраций. Библиографический список содержит 18 наименований, так же 1 публикацию соискателя.
