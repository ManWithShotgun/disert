\chapter*{ОБЩАЯ ХАРАКТЕРИСТИКА РАБОТЫ}
\addcontentsline{toc}{chapter}{ОБЩАЯ ХАРАКТЕРИСТИКА РАБОТЫ} % in Content
\textbf{Связь работы с научными программами (проектами) и темами}\\

Диссертационная работа связана с тематикой НИР, выполняемых в рамках научно-исследовательского направления кафедры «Информационные технологии» Гомельского государственного технического университета им. П. О. Сухого.
Тема диссертации соответствует приоритетным направлениям
фундаментальных исследований в Республики Беларусь. Диссертационная
работа выполнялась в период с 2017 по 2019 годы в рамках отдельного
подзадания по государственной программе научных исследований
«Конвергенция-2020», номер гос. регистрации 20162284.
\\

\textbf{Цель и задачи исследования}\\

Целью работы является создание системы определяющий возможность рождения нового резонанса нейтрального спина 1 (${Z}^{\prime}$) 
из доступных данных групп \textit{ATLAS} для ${W}^{+}{W}^{-}$ распадов. В качестве результатов работы будут получены 
ограничения на соответствующие $Z$-${Z}^{\prime}$-коэффициенты смешивания и на массу $M_{Z^\prime}$.
Для достижения поставленной цели были поставлены следующие задачи:

\begin{itemize}
	\item[--] Изучить процесс ${W}^{+}{W}^{-}$ распадов и рождения ${Z}^{\prime}$ бозонов на БАК;
	
	\item[--] Описание и создание математической модели изученного процесса;
	
	\item[--] Создание программного обеспечения для вычисления ограничений на соответствующие $Z$-${Z}^{\prime}$-коэффициенты смешивания и на массу $M_{Z^\prime}$;
	
	\item[--] Построение \textit{web}-приложения для демонстрации результатов;
	
\end{itemize}

Изучение появления электрослабых бозонов дает мощную проверку спонтанного нарушения 
калибровочной симметрии стандартной модели и может быть использовано для поиска новых явлений за пределами стандартной модели. 
Дополнительные нейтральные векторные бозоны ${Z}^{\prime}$, распадающиеся на заряженные пары калибровочных векторных бозонов ${W}^{+}{W}^{-}$, 
прогнозируются во многих сценариях новой физики, включая модели с расширенным калибровочным сектором.
\\

\vspace{16pt}
\textbf{Научная новизна}\\

Научная новизна работы заключается в том, что впервые получены ограничения на угл смешивания ${Z}^{\prime}$-бозонов в
процессе рождения ${W}^{+}$${W}^{-}$ пар в протон-протнных столкновениях для светимостей 1000 фб${}^{−1}$ и 3000 фб${}^{−1}$ на Большом Адронном коллайдер, а также создан программный модуль, позволяющий выполнять: имитационное моделирование рождения ${Z}^{\prime}$ в процессе ${W}^{+}{W}^{-}$
на Большом Адронном коллайдер с учётом эффектов $Z$-${Z}^{\prime}$ смешивания.
\\

\textbf{Положения, выносимые на защиту}\\

Автором защищаются:
\begin{itemize}
	\item[--] имитационная модель процесса рождения ${Z}^{\prime}$-бозонов в протон-протнных столкновениях с учетом эффектов $Z$-${Z}^{\prime}$ смешивания;
	
	\item[--] программный модуль для имитационного моделирования процесса
	рождения ${Z}^{\prime}$-бозонов в протон-протнных столкновениях с учетом эффектов $Z$-${Z}^{\prime}$ смешивания в условиях
	выполненных экспериментов на \textit{ATLAS};
	
	\item[--] рассчитаные ограничения на углы смешивания ${Z}^{\prime}$-бозонов в
	процессе рождения ${W}^{+}$${W}^{-}$ пар в протон-протнных столкновениях
	в условиях выполненных экспериментов на Большом Адронном коллайдер, а так же рассчитаные ограничения для светимостей 1000 фб${}^{−1}$ и 3000 фб${}^{−1}$.
	
\end{itemize}
\\

\textbf{Личный вклад соискателя}\\

Научные и практические результаты диссертации, положения, выносимые на защиту, разработаны и получены лично соискателем или при его непосредственном участии.
\\

\textbf{Апробация результатов диссертации}\\

Результаты работы докладывались на cтуденческой международной научно-практической конференции «Научное сообщество студентов XXI столетия. ТЕХНИЧЕСКИЕ НАУКИ» номер LXXIX.
\\

\vspace{5cm}
\textbf{Опубликованность результатов диссертации}\\

Результаты диссертационных исследований, связанных с измерением процесса рождения ${W}^{+}{W}^{-}$ пар в протон-протонных столкновениях и получены экспериментальные ограничения на угол смешивания ${Z}^{\prime}$-бозонов  ожидают публикаций.
\\

\textbf{Структура и объем диссертации}\\

Диссертационная работа состоит из введения, четырёх глав, заключения и библиографического списка. Объем диссертации – 75 листов, включая 3 приложения и 46 иллюстраций. Библиографический список содержит 18 наименований, так же 1 публикацию соискателя.
