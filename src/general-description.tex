\chapter*{ОБЩАЯ ХАРАКТЕРИСТИКА РАБОТЫ}
\addcontentsline{toc}{chapter}{ОБЩАЯ ХАРАКТЕРИСТИКА РАБОТЫ} % in Content
\textbf{Связь работы с научными программами (проектами) и темами}

Работа проводилась в рамках научно-исследовательского направления кафедры «Информационные технологии» Гомельского государственного технического университета им. П. О. Сухого.


\textbf{Цель и задачи исследования}

Целью работы является создание системы определяющий возможность рождения нового резонанса нейтрального спина 1 (${Z}^{\prime}$) 
из доступных данных групп \textit{ATLAS} и \textit{CMS} для ${W}^{+}{W}^{-}$ распадов. В качестве результатов работы будут получены 
ограничения на соответствующие $Z$-${Z}^{\prime}$-коэффициенты смешивания и на массу $M_{Z^\prime}$.
Для достижения поставленной цели были поставлены следующие задачи:

\begin{itemize}
	\item[--] Изучить процесс ${W}^{+}{W}^{-}$ распадов и рождения ${Z}^{\prime}$ бозонов на БАК;
	
	\item[--] Описание и создание математической модели изученного процесса;
	
	\item[--] Создание программного обеспечения для вычисления ограничений на соответствующие $Z$-${Z}^{\prime}$-коэффициенты смешивания и на массу $M_{Z^\prime}$;
	
	\item[--] Построение \textit{web}-приложения для демонстрации результатов;
	
\end{itemize}

Изучение появления электрослабых бозонов дает мощную проверку спонтанного нарушения 
калибровочной симметрии стандартной модели и может быть использовано для поиска новых явлений за пределами стандартной модели. 
Дополнительные нейтральные векторные бозоны ${Z}^{\prime}$, распадающиеся на заряженные пары калибровочных векторных бозонов ${W}^{+}{W}^{-}$, 
прогнозируются во многих сценариях новой физики, включая модели с расширенным калибровочным сектором.

\textbf{Положения, выносимые на защиту}

Автором защищаются:
\begin{itemize}
	\item[--] методика моделирования процесса ${W}^{+}{W}^{-}$ распадов и рождения ${Z}^{\prime}$ бозонов;
	
	\item[--] методика ;
	
	\item[--] методика . 
	
\end{itemize}

\textbf{Личный вклад соискателя}

Научные и практические результаты диссертации, положения, выносимые на защиту, разработаны и получены лично соискателем или при его непосредственном участии.

\textbf{Апробация результатов диссертации}



\textbf{Опубликованность результатов диссертации}

Результаты диссертационных исследований, связанных с измерением процесса рождения ${W}^{+}{W}^{-}$ пар в протон-протонных столкновениях и получены экспериментальные ограничения на угол смешивания ${Z}^{\prime}$-бозонов  ожидают публикаций.

\textbf{Структура и объем диссертации}

Диссертационная работа состоит из введения, четырёх глав, заключения и библиографического списка. Объем диссертации – 75 листов, включая 3 приложения и 46 иллюстраций. Библиографический список содержит 18 наименований, так же 1 публикацию соискателя.
