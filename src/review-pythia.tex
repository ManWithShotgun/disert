\textit{PYTHIA} это программа для генерации событий физики
высоких энергий, т.е., для описания столкновений таких
высокоэнергетических элементарных частиц, как электрон,
позитрон, протон и антипротон в различных комбинациях.
Информация для моделирования взята по большей части из
собственных исследований в ЦЕРН, однако, много формул и
другой информации почерпнуто из научной литературы~\cite{review-pythia}.

С 1997 года по нынешнее время использовалась версия
этого Монте-Карло генератора, написанная в \textit{FORTRAN77}
(текущая версия 6.4). Сейчас программа переписана в \textit{С++}
(версия 8.1), однако, до тех пор, пока не осуществлен
перевод всех возможностей, обе версии используются и
поддерживаются одновременно.


Назначение генераторов физических событий:
\begin{itemize}
	\item[--] Дают физикам представление о типе событий, которые
	они надеются увидеть, и об их скорости набора;
	\item[--] Помогают в планировании новых детекторных установок,
	то есть, оптимизировать их характеристики для изучения
	интересующих сценариев физических событий в рамках
	существующих ограничений;
	\item[--] Являются инструментом для проработки стратегии
	анализа данных (оптимизации отношения “сигнал/шум”);
	\item[--] Используются в качестве метода оценки коррекций на
	геометрические и кинематические ограничения области
	чувствительности (acceptance) детекторов;
	\item[--] Используются в качестве удобной рабочей оболочки для
	интерпретации наблюдаемых феноменов в терминах
	Стандартной Модели.
\end{itemize}

Квантовая механика вносит концепцию случайности в
поведение физических процессов. Достоинством
генераторов событий (\textit{event generators}) является то, что эта
случайность может быть смоделирована при помощи метода
Монте-Карло. Сущность метода заключается в том, что, вопервых подразумевается наличие генератора
псевдослучайных чисел, т.е. функции, которая при вызове
возвращает число R в пределах от 0 до 1, при этом
распределение R является плоским, и с достаточной
точностью значения \textit{R} являются нескоррелированными~\cite{review-pythia}.
Затем эти значения \textit{R} используется для розыгрыша сценария
конкретного цикла события (выбор конкретного значения
для различных известных распределений величин, выбор
времени распада и т.п.) Разыграв статистически достаточное
количество событий, мы можем построить интересующие
нас распределения (например, диапазон энергий для
продуктов интересующего механизма реакции).
Что касается упомянутых известных распределений,
то, например, дифференциальное сечение реакции
рассчитывается из кинематических соотношений при
введении матричного элемента (известного, либо
предложенного теоретиками исходя из перспективных
моделей), для учета высших порядков КХД вводится
значение дополнительного параметра (K-множителя).
Следует заметить, что при генерации значительного
числа событий (миллионов), становится актуальной
проблема скоррелированости псевдослучайных чисел, и
вместо встроенных в \textit{C++ Random} генераторов приходится
применять специально созданные программы.


С точки зрения описания физики событий полная
процедура генерации события разделяется на 3 стадии:

\begin{enumerate}
	\item Генерация «процесса», который определяет природу
	события. Зачастую это могут быть «жесткие процессы»,
	такие как $gg \rightarrow {h}^{0} \rightarrow ZZ \rightarrow {m}^{+}{m}^{-}{qq}_{bar}$ (а также другие
	процессы), которые могут быть просчитаны в рамках
	теории малых возмущений.

	\item Генерация всех подчиненных процессов на партонном
	уровне, включая гамма-излучение, многократное партонные
	взаимодействия и структуру непровзаимодействовавшего
	пучка. Такие феномены приблизительно описываются
	теорией малых возмущений, однако непертурбативные
	поправки уже существенны.
	\item Адронизация этой партонной конфигурации
	(фрагментация струй, распады нестабильных частиц).
	Только феноменологическое описание
\end{enumerate}

Этим стадиям отвечают три класса \textit{ProcessLevel},
\textit{PartonLevel} и \textit{HaronLevel}, соответственно. Классы: \textit{Event}
(члены класса \textit{process} и \textit{event}), \textit{BeamParticles}, база данных
\textit{Settings}

С точки зрения технического устройства,
взаимодействия пользователя и генератора проявляется в
трех фазах:


\begin{enumerate}
	\item Инициализация, когда формулируется задание.
	\item Генерация индивидуального события (цикл события).
	\item Вывод окончательной статистики.
\end{enumerate}

Программа содержит теорию и модели для ряда
аспектов физики, включая так называемые мягкие и жесткие
взаимодействия, распределения партонов, партонные струи
начального и конечного состояний, многократные
партонные взаимодействия, фрагментацию и распады~\cite{review-pythia}.

Встроенные \textit{C++} методы программы обеспечивают
доступ к информации как об отдельной частице либо
процессе на любом этапе розыгрыша события, так и о
событии в целом. Встроенные средства вывода позволяют
получить статистическую информацию и гистограммы в
виде \textit{ASCII} кода (который можно сохранить в файл для
дальнейшего использования).

Пакет \textit{Pythia} является компактной (\textit{5 Мб} )
независимой программой, поэтому несложно скачать и
установить себе собственную локальную версию. Для этого
архив установочный код (в 2011 году был \textit{8145.tgz}) можно
скачать с сайта разработчика или с кафедрального сайта~\cite{review-pythia}.
Затем его нужно распаковать (команда \textit{tar -xzf 8145.tgz})
и скомпилировать (команда make в распакованной корневой
директории, занимает ~2 минут на \textit{lx}).

Быстрее всего начать работу с обучающими и своими
первыми скриптами в поддиректории examples. В пакеты \textit{Pythia} можно создать такой объемный процесс, как генератор
моделирющий столкновение барионов на \textit{LHC} с рождением
топ-кварков. Так же в на сайте разработчиков бибилиотеки есть руководство в котором описываются возможности по настройке
свойств процессов и получения информации (например,
можно сделать бозон ${Z}^0$ стабильной частицей инструкцией \textit{pythia.readString("23:onMode=off")}). 

%http://lib.sinp.msu.ru/static/tutorials/141_Leontiev_Zadahi_2011.pdf