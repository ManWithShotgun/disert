В настоящей пояснительной записке применяются следующие термины, обозначения и сокращения.

БАК -- Большой Адронный Коллайдер.

ДЯ -- Дрелл-Янга.

КХД -- Квантовая хромодинамика, калибровочная теория сильных взаимодейсвий.

ЛЭП -- большой электрон-позитронного коллайдер.

НФ -- Новая Физика.

СМ -- Стандартная Модель.

CУБД -- Система управления базами данных

\textit{ATLAS} (\textit{A Toroidal LHC ApparatuS}) -- один из четырёх основных экспериментов на Большом адронном коллайдере в Европейской организации ядерных исследований \textit{CERN} в городе Женева (Швейцария).

\textit{AWS} (\textit{Amazon Web Services}) -- коммерческое публичное облако, поддерживаемое и развиваемое компанией \textit{Amazon}.

\textit{CMS} (\textit{Compact Muon Solenoid}) -- один из двух больших универсальных детекторов элементарных частиц на Большом адронном коллайдере.

\textit{HCI} (\textit{Human-Computer Interaction}) -- взаимодействие человека и компьютера.

\textit{JVM} (\textit{Java Virtual Machine}) -- виртуальная машина \textit{Java} — основная часть исполняющей системы Java.

\textit{JIT} (\textit{Just-In-Time compiler}) -- часть виртуальной машины \textit{Java}, которая используется для ускорения выполнения приложений.

\textit{JMX} (\textit{Java Management Extensions}) -- технология \textit{Java}, предназначенная для контроля и управления приложениями.

\textit{LEP} (\textit{Large Electron-Positron collider}) -- ускоритель заряженных частиц осуществляющий столкновения электронов и их античастиц -- позитронов.

\textit{LHC} (\textit{Large Hadron Collider}) --  ускоритель заряженных частиц на встречных пучках, предназначенный для разгона протонов и тяжёлых ионов и изучения продуктов их соударений.

\textit{MVC} (\textit{Model-View-Controller}) -- схема разделения данных приложения, пользовательского интерфейса и управляющей логики на три отдельных компонента.

\textit{NLO} (\textit{Next-to-leading order}) -- возмущающие вычисления КХД следующего порядка.

\textit{SLC} (\textit{Stanford Linear Collider}) -- линейный коллайдер сталкивающий электроны и позитроны каждый с энергией до 50 ГэВ.

\textit{SMC} (\textit{Shower Monte Carlo}) -- метод моделирования событий в физике высоких энергий при помощи генератора случайных величин.

\textit{SSM} (\textit{Supersymmetric Standard
Model}) -- идея суперсимметрии в стандартной модели -- симметрии между фермионами и бозонами. 

\textit{SVG} (\textit{Scalable Vector Graphics}) -- язык разметки масштабируемой векторной графики предназначен для описания двумерной векторной и смешанной векторно или растровой графики в формате \textit{XML}.

\textit{UI} (\textit{User Interface}) -- коммуникационный канал, по которому осуществляется взаимодействие пользователя и компьютера.

