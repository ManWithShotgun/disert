В настоящей пояснительной записке применяются следующие термины, обозначения и сокращения.

БАК -- Большой Адронный Коллайдер.

ДЯ -- Дрелл-Янга.

КХД -- Квантовая хромодинамика, калибровочная теория сильных взаимодейсвий.

ЛЭП -- большой электрон-позитронного коллайдер.

НФ -- Новая Физика.

СМ -- Стандартная Модель.

CУБД -- Система управления базами данных

ATLAS (A Toroidal LHC ApparatuS) -- один из четырёх основных экспериментов на Большом адронном коллайдере в Европейской организации ядерных исследований CERN в городе Женева (Швейцария).

AWS (Amazon Web Services) -- коммерческое публичное облако, поддерживаемое и развиваемое компанией Amazon.

CMS (Compact Muon Solenoid) -- один из двух больших универсальных детекторов элементарных частиц на Большом адронном коллайдере.

HCI (Human-Computer Interaction) -- взаимодействие человека и компьютера.

JVM (Java Virtual Machine) -- виртуальная машина Java — основная часть исполняющей системы Java.

JIT (Just-In-Time compiler) -- часть виртуальной машины Java, которая используется для ускорения выполнения приложений.

JMX (Java Management Extensions) -- технология Java, предназначенная для контроля и управления приложениями.

LEP (Large Electron-Positron collider) -- ускоритель заряженных частиц осуществляющий столкновения электронов и их античастиц -- позитронов.

LHC (Large Hadron Collider) --  ускоритель заряженных частиц на встречных пучках, предназначенный для разгона протонов и тяжёлых ионов и изучения продуктов их соударений.

MVC (Model-View-Controller) -- схема разделения данных приложения, пользовательского интерфейса и управляющей логики на три отдельных компонента.

NLO (Next-to-leading order) -- возмущающие вычисления КХД следующего порядка.

SLC (Stanford Linear Collider) -- линейный коллайдер сталкивающий электроны и позитроны каждый с энергией до 50 ГэВ.

SMC (Shower Monte Carlo) -- метод моделирования событий в физике высоких энергий при помощи генератора случайных величин.

SSM (Supersymmetric Standard
Model) -- идея суперсимметрии в стандартной модели -- симметрии между фермионами и бозонами. 

SVG (Scalable Vector Graphics) -- язык разметки масштабируемой векторной графики предназначен для описания двумерной векторной и смешанной векторно или растровой графики в формате XML.

UI (User Interface) -- коммуникационный канал, по которому осуществляется взаимодействие пользователя и компьютера.

