\chapter*{ВВЕДЕНИЕ}
\addcontentsline{toc}{chapter}{ВВЕДЕНИЕ} % in Content
Одной из основных задач современной теоретической и экспериментальной физики является проверка Стандартной модели электрослабых и сильных взаимодействий элементарных частиц (СМ) [1-15], которая осуществлялась в ускорительных экспериментах на высокоэнергетических коллайдерах, таких как \textit{LEP}, \textit{SLC}, \textit{Tevatron}, \textit{HERA} и др., а также интенсивно ведется в настоящее время на Большом адронном коллайдере \textit{LHC}. Последний громкий успех СМ связан с открытием хиггсовского бозона в экспериментах CMS и \textit{ATLAS} на \textit{LHC}. Для более детального исследования свойств хиггсовсого бозона планируются новые коллайдерные эксперименты, такие как проекты \textit{ILC} и \textit{CLIC}. Стандартная модель не объясняет, что такое гравитация и как она связана с другими силами и частицами. Также она не объясняет, почему основными частицами вещества являются кварки и лептоны и сколько их должно быть. Кроме этого Стандартная модель не объясняет таких явлений, которые по праву должны учитываться при больших энергиях, а теперь исследуются ускорителями частиц. Одно их таких явлений – «темная материя». По последним данным считается, что доминирующей формой материи во Вселенной является так называемая «Темная материя». Без темной материи галактики и звезды не сформировались бы и жизни не существовало бы. Только в последние 10-15 лет ученые добились существенного прогресса в понимании свойств темной материи. Недавние наблюдения влияния темной материи на структуру Вселенной показали, что она отличается от любой формы материи, которую обнаружили или измерили в лаборатории. В то же время появились новые теории, которые могут сказать нам, что такое темная материя. В настоящее время на современных ускорителях элементарных частиц ведутся поиски кандидатов на частицы темной материи. Если эти частицы имеют массы, которые измеряются в шкале ТэВ, то они могут быть обнаружены на Большом адронном коллайдере. Однако проверка того, что эти новые частицы действительно связаны с темной материей, потребует, получение их характеристик.