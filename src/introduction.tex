\chapter*{ВВЕДЕНИЕ}
\addcontentsline{toc}{chapter}{ВВЕДЕНИЕ} % in Content
Одной из основных задач современной теоретической и экспериментальной физики является проверка Стандартной модели электрослабых и сильных взаимодействий элементарных частиц (СМ)~\cite{2part-1}, которая осуществлялась в ускорительных экспериментах на высокоэнергетических коллайдерах, таких как \textit{LEP}, \textit{SLC}, \textit{Tevatron}, \textit{HERA} и др.~\cite{sirunyan:2017}, а также интенсивно ведется в настоящее время на Большом адронном коллайдере \textit{LHC}~\cite{main-book}. Последний громкий успех СМ связан с открытием хиггсовского бозона в экспериментах CMS и \textit{ATLAS} на \textit{LHC}~\cite{Krasnikov:2004}. Для более детального исследования свойств хиггсовсого бозона планируются новые коллайдерные эксперименты, такие как проекты \textit{ILC} и \textit{CLIC}~\cite{2part-pankov}. Стандартная модель не объясняет, что такое гравитация и как она связана с другими силами и частицами. Также она не объясняет, почему основными частицами вещества являются кварки и лептоны и сколько их должно быть. Кроме этого Стандартная модель не объясняет таких явлений, которые по праву должны учитываться при больших энергиях, а теперь исследуются ускорителями частиц. Одно их таких явлений – «темная материя». По последним данным считается, что доминирующей формой материи во Вселенной является так называемая «Темная материя». Без темной материи галактики и звезды не сформировались бы и жизни не существовало бы. Только в последние 10-15 лет ученые добились существенного прогресса в понимании свойств темной материи. Недавние наблюдения влияния темной материи на структуру Вселенной показали, что она отличается от любой формы материи, которую обнаружили или измерили в лаборатории. В то же время появились новые теории, которые могут сказать нам, что такое темная материя. В настоящее время на современных ускорителях элементарных частиц ведутся поиски кандидатов на частицы темной материи. Если эти частицы имеют массы, которые измеряются в шкале ТэВ, то они могут быть обнаружены на Большом адронном коллайдере. Однако проверка того, что эти новые частицы действительно связаны с темной материей, потребует, получение их характеристик~\cite{nuclphys:weak}.

Экспериментальные программы физических экспериментов уже завершенных коллайдерных исследований (\textit{LEP}, \textit{TEVATRON}, и др.), текущих экспериментов (Большой адронный коллайдер -- Швейцария, Франция), строящихся (\textit{NICA} -- Россия) и планируемых в ближайшем будущем экспериментов (Международный линейный электрон-позитронный коллайдер в Японии; \textit{CLIC} -- Швейцария, Франция) содержат разделы, посвящённые исследованиям моделей новой физики, выходящим за рамки Стандартной модели элементарных частиц. В том числе, эти программы физических экспериментов включают в себя задачи по поиску эффектов новых ${Z}^{\prime}$-бозонов и получению ограничений на параметры ${Z}^{\prime}$-бозонов. Поэтому задача оценки ограничений на углы смешивания и массу ${Z}^{\prime}$-бозонов в условиях экспериментов на Большом адронном коллайдере является актуальной.

В силу вероятностной природы процессов взаимодействия элементарных частиц, имитационное моделирование является одним из главных инструментов для моделирования результатов текущих и планируемых экспериментов в физике элементарных частиц и высоких энергий. В настоящее время, имитационное моделирование широко применяется для моделирования фоновых событий для очистки экспериментальных данных а также для моделирования эффектов новой физики.

В настоящей работе имитационное моделирование использовано для моделирования  эффектов $Z-{Z}^{\prime}$ смешивания в процессе $pp \rightarrow W^+W^- + X$ в условиях экспериментов на Большом адронном коллайдере и анализа экспериментальных данных Большого адронного коллайдера с целью получения оценок (ограничений) на параметры ${Z}^{\prime}$-бозонов: углы смешивания и массу ${Z}^{\prime}$-бозонов.