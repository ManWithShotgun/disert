Возмущающие вычисления КХД следующего порядка (NLO), а также ливень Монте-Карло
(SMC) программы являются фундаментальными инструментами современной феноменологии физики элементарных частиц. В частности, программы SMC включают описание общего адронного столкновения высокой энергии
процесс, начиная от столкновения между составляющими и развития партонного ливня, что
увеличивает количество частиц в конечном состоянии за счет сильно упорядоченных последующих выбросов.
В конце концов, интерфейс с феноменологической моделью адронизации позволяет сравнивать
с экспериментальными данными. По этим причинам они обычно используются экспериментаторами для
моделировать сигнальные и фоновые процессы в физических поисках. Тем не менее, спрос на
лучшие и лучшие прогнозы из экспериментов с высокой энергией требуют повышения точности
существующих SMC, включая исправления NLO. Метод MC @ NLO [1] сначала показал, как
достичь точности NLO для инклюзивных количеств, реализуя жесткий подпроцесс в NLO и
развитие ливней в ведущем логарифмическом приближении, избегая двойного счета
излучение. Таким образом можно достичь преимуществ обоих подходов: генерация исключительных конечных состояний
SMC и точность расчетов NLO.
