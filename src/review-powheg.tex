Возмущающие вычисления КХД следующего порядка \textit{Next-to-leading order} (\textit{NLO}), а также \textit{Shower Monte Carlo}
(\textit{SMC}) программы являются фундаментальными инструментами современной феноменологии физики элементарных частиц. В частности, программы \textit{SMC} включают описание общего адронного столкновения высокой энергии
процесс, начиная от столкновения между составляющими и развития партонного потока, что
увеличивает количество частиц в конечном состоянии за счет сильно упорядоченных последующих выбросов~\cite{review-powheg}.
В конце концов, интерфейс с феноменологической моделью адронизации позволяет сравнивать
с экспериментальными данными. По этим причинам они обычно используются экспериментаторами для
моделирования сигнальных и фоновых процессов в физических поисках. Тем не менее, спрос на
лучшие и лучшие прогнозы для экспериментов с высокой энергией требуют повышения точности
существующих \textit{SMC}, включая исправления \textit{NLO}. Метод \textit{MC@NLO}~\cite{review-powheg} сначала показал, как
достичь точности \textit{NLO} для инклюзивных количеств, реализуя жесткий подпроцесс в \textit{NLO} и
развитие ливней в ведущем логарифмическом приближении, избегая двойного счета
излучений. Таким образом можно достичь преимуществ обоих подходов: генерация исключительных конечных состояний
\textit{SMC} и точность расчетов \textit{NLO}.

Метод \textit{POWHEG} - это другое предписание для сопряжения вычислений \textit{NLO} с партоном
душевые генераторы.
Этот метод не зависит от программы Монте-Карло, используемой для последующего принятия потока частиц и
генерирует только положительные взвешенные события. В этом отношении это улучшает подход \textit{MC@NLO}.
До сих пор метод \textit{POWHEG} был успешно применен к нескольким процессам, как на
лептонные и адронные коллайдеры~\cite{review-powheg}. В этих реализациях это
подключен к программам \textit{SMC HERWIG}, \textit{PYTHIA} и \textit{HERWIG ++}.

В методе \textit{POWHEG} самое сильное излучение
генерируется первым, независимо от
следующие. Схематически
самое жесткое излучение распределяется в соответствии с

\begin{equation} \label{eq:1-1} 
	d\sigma = \bar{B} ({\Phi}_{B}) d{\Phi}_{B}[{\Delta}_{R}({p}_{T}^{min}) + \frac{R({\Phi}_{R})}{B({\Phi}_{B})}{\Delta}_{R}({k}_{T}({\Phi}_{R}))d{\Phi}_{rad}],
\end{equation}

\begin{flushleft}
	где $B({\Phi}_{B})$ -- вклад Борна;\\
	$\bar{B}({\Phi}_{B}) = B({\Phi}_{B}) + [V({\Phi}_{B}) + \int d{\Phi}_{rad}R({\Phi}_{R})]$ -- является дифференциальным сечением \textit{NLO} при фиксированной основной кинематике Борна и интегрированной по
	радиационные переменные.
\end{flushleft}

Поперечный импульс испускаемого партона относительно
пучка или другой частицы, в зависимости от особенности области, обозначается через ${k}_{T}({\Phi}_{R})$.
нижний предел ${p}_{T}^{min}$
необходимо, чтобы константа связи не достигала нефизических значений.
$V({\Phi}_{B})$ и $R({\Phi}_{R})$ являются виртуальными и действительными поправками и в выражении внутри
квадратной скобки в формуле (2) процедура, которая заботится об отмене мягких и коллинеарных
особенностей, например, \textit{Frixione-Kunszt-Signer} (\textit{FKS}) или Катани-Сеймур (\textit{CS})
дипольное вычитание. затем,

\begin{equation} \label{eq:1-2}  
	{\Delta}_{R}({P}_{T}) = exp[-\int d{\Phi}_{rad}\frac{R({\Phi}_{R})}{B({\Phi}_{B})}\theta({k}_{T}({\Phi}_{R}) - {p}_{T})]
\end{equation}

Это \textit{POWHEG S}, то есть вероятность того, что выброс не будет тяжелее, чем ${p}_{T}$. Уравнение~\ref{eq:1-1} можно рассматривать как улучшение исходной формулы для наиболее сложных выбросов \textit{SMC}, поскольку
сечение Борна заменяется на $\bar{B}({\Phi}_{B})$, которое по построению нормированы на \textit{NLO}.
При малых поперечных импульсах \textit{POWHEG} \textit{CS} становится равным стандартному \textit{SMC}.
Тем не менее, \textit{NLO}
область излучения с высоким ${p}_{T}$ правильно описывается реальными вкладом:

\begin{equation} \label{eq:1-3}   
	d\sigma \approx \bar{B}({\Phi}_{B}) d{\Phi}_{B}\frac{R({\Phi}_{R})}{B({\Phi}_{B})}d{\Phi}_{rad}\approx R({\Phi}_{R})d{\Phi}_{B}d{\Phi}_{rad}
\end{equation}


Поскольку ${\Delta}_{R} \approx 1$ и $\bar{B}/B \approx 1 + \Theta ({\alpha}_{s})$ после генерации самого жесткого излучения можно
использовать интерфейс с любым доступным генератором столкновений, для того, чтобы обработать остальной поток,
чтобы избежать двойной регистрации частиц, \textit{SMC} должен быть либо ${p}_{T}$-упорядоченным, либо иметь возможность
наложить вето на выбросы с ${p}_{T}$ сложнее, чем первый.

В реальном процессе столкновения присутствуют несколько цветных безмассовых партонов, либо в начальном, либо в
конечное состояние. Таким образом, следует повторить процедуру, изложенную в начале главы для каждого возможного единственного числа
областей, связанных с любой безмассовой цветной веткой, становящейся коллинеарной к другой, или мягкой.
Для этого всё реальное сечение эмиссии раскладывается в сумму слагаемых, каждое
из которых имеет не более одной коллинеарной и одной мягкой особенности. Затем излучение генерируется
независимо в каждом из этих регионов, но сохраняется только самое сильное излучение, и событие
генерируется в соответствии с ароматом и кинематикой, связанной с ним. Из-за этой сложности,
автоматический инструмент, \textit{POWHEG-BOX}, был построен~\cite{review-powheg}, чтобы помочь включению новых
процессов. С другой стороны, \textit{POWHEG-BOX} также может рассматриваться как библиотека, где ранее
реализованные процессы доступны в общей структуре. Процессы реализованы так
далеко и уже доступны в публичной версии включают: $W$, $Z / y$ производство одного вектора бозона,
Бозон Хиггса через глюон и вектор бозон-фьюжн, однолучевой в \textit{s}- и \textit{t}-каналы.

Пользователь, желающий включить новый расчет \textit{NLO}, должен знать только, как сообщить
нужную информация для \textit{POWHEG-BOX}. Это происходит либо путем определения соответствующих переменных,
либо предоставляя необходимые процедуры Фортрана. Требуемые входы:

\begin{enumerate}
	\item Количество ветвей в процессе Борна, например, \textit{nlegborn} = 5 для $pp \rightarrow (Z \rightarrow {e}^{+}{e}^{-}) j$
	\item Список ароматов \textit{Born} и \textit{Real}, согласно соглашениям \textit{PDG}~\cite{review-powheg}, аромат
	определен входящий (исходящий) для входящих (исходящих) фермионных линий, например, для $bu \rightarrow Ztsg$.
	\item Процедура Борновского фазового пространства, которая, учитывая случайные числа в единицах измерения \textit{ndims}
	гиперкуба, задайет борновское фазовое пространство якобиана и возвращает импульсы в неизвестных \textit{х}.
	\item Подпрограммы, выполняющие инициализацию соединений и настройку
	шкалы факторизации и перенормировки.
	\item Процедура амплитуды Борна в квадрате, для заданного набора импульсов и ароматов
	конфигурации, возвращает $B = {|M|}^{2}$
	суммируется и усредняется по цвету и спирали
	как упорядоченные по цвету квадраты Борна, амплитуды ${B}_{jk}$ и спиральность коррелировали по квадрату Борна
	амплитуды ${B}_{k,\mu\nu}$, где \textit{k} пробегает все внешние глюоны.
	\item Подпрограмма квадрата амплитуды реального излучения, которая возвращает \textit{R} для заданных импульсов и
	список ароматов.
	\item Конечная часть интерференции борновского и виртуального амплитудных вкладов ${\nu}_{b} = 2Re\{B\times V\}$ после вычета общего множителя $N = \frac{{(4p)}^\xi}{G(1-\xi)}({\frac{{\mu}_{R}^{2}}{{Q}^{2}}}^{\xi})$. Эта рутина определяется импульсами и списком ароматов в качестве входных данных.
	\item Цветовые структуры Борна в большом пределе ${N}_{c}$ задаются через интерфейс \textit{Les Houches}~\cite{review-powheg}.
\end{enumerate}

Пункты (1-7) являются обычными ингредиентами, необходимыми для выполнения расчета \textit{NLO}
в любом методе вычитания. Пункт 8 необходим для обеспечения определенной цветовой структуры
генератор \textit{SMC}. Внутри \textit{POWHEG-BOX} реализована процедура вычитания \textit{FKS}. В начале пакет автоматически оценивает комбинаторику, выявляя
все особые области и соответствующие базовые вклады Борна. Он также выполняет
проекцию реальных вкладов на особую область и вычисляет вычитание
контртермы из мягких и коллинеарных приближений реальных выбросов. Затем пакет строит
\textit{ISR} и \textit{FSR} фазовые пространства, согласно \textit{FKS} параметризации особой области и
выполняет интеграцию. В конце концов, каждый получает дифференциальное сечение \textit{NLO}. На данном этапе,
можно также взаимодействовать с некоторой процедурой анализа, чтобы получить дифференциальные распределения \textit{NLO} как
побочный продукт. После этапа интеграции выполняется вычисление верхних границ для
эффективная генерация событий, подавленных эффектом Судакова, а затем генерация сильнейшего излучения. На данный момент генерируется события, которые содержат не более одного излучения, которое должны быть переданы в стандартную программу \textit{SMC}, для разработки остальных.

%https://indico.desy.de/indico/event/1964/session/16/contribution/238/material/0/0.pdf
