Наиболее яркими примерами генераторов событий являются очень успешные, хорошо отлаженные программы
PYTHIA [1] и HERWIG [2]. Они были построены за последние десятилетия наряду с экспериментальными открытиями, и большинство особенностей, видимых в прошлых и настоящих экспериментах, могут быть описаны
ими. Тем не менее, необходимость в более высокой точности для решения задач новых энергетических масштабов происходящих
на \textit{LHC}, сложность конечных состояний в этих масштабах, необходимость обслуживания и желание легкой возможности реализовать новые физические модели, требовалось переписать эти пакеты на современном языке программирования, обеспечивающем более высокий уровень модульности. Объектно-ориентированные рамки отвечают последним
требования и в связи с предпочтениями сообщества к \textit{С++}, новое поколение генераторов событий
построено на этом языке программирования. Это привело к улучшению повторных реализаций в форме
программ \textit{PYTHIA 8} [3] и \textit{HERWIG ++} [4] - преемники версий написаных на \textit{Fortran}, упомянутых выше
и к созданию генератора событий \textit{SHERPA} [5].
В связи с этим в последнее десятилетие стали доступны пакеты для расчетов с опережающим порядком. Яркими примерами являются \textit{MCFM} [6] и \textit{NLOJET ++} [7]. Соответствующие методы реализованы, например, в \textit{MC@NLO} [10], который основан
на фортранской версии \textit{HERWIG}, в \textit{HERWIG ++} [11] и в некоторых более специализированных программах [12].
Тем не менее, полные расчеты следующего за ведущим порядком, лежащие в основе этих новых методов, очень сложны, и до сегодняшнего дня контролируются только процессы с пятью внешними ветвями. Но
с другой стороны, многие важные экспериментальные сигнатуры зависят от конечных состояний с более высокой кратностью, что
инициировало существенную деятельность по совершенствованию методов и инструментов с точностью на уровне множества ветвей, так что
теперь доступно несколько пакетов, которые могут вычислять соответствующие сечения и генерировать события
полностью автоматизированным способм. Наиболее яркими примерами являются \textit{ALPGEN} [13], \textit{CompHEP} / \textit{CalcHEP} [14],
\textit{HELAC-PHEGAS} [15], \textit{MADGRAPH} [16], \textit{WHIZARD} [17] и \textit{AMEGIC ++} [18]. В настоящее время только библиотека \textit{AMEGIC ++} реализует данный функционал и встроенная в полноценный генератор событий, а именно в среду \textit{SHERPA}. Чтобы
преобразовать многочастичные события на уровне партонов, которые предоставляются этими инструментами в ведущем порядке, в
события уровня адронов, было разработано несколько алгоритмов, все из которых направлены на сохранение логарифмической
точности партонного потока и дополнения его точным результатом возмущающего ведущего порядка для
учитывания кратности струи.

\textit{SHERPA} [5] является аббревиатурой от «“Simulation of High Energy Reactions of Particles». Программа является
полная структура генерации событий, которая была построена с нуля и полностью написана в
современный объектно-ориентированный язык программирования \textit{C++}

Построение SHERPA осуществлялось способом, в значительной степени определяемым следующими тремя парадигмами:


% https://arxiv.org/format/0811.4622