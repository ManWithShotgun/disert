Наиболее яркими примерами генераторов событий являются очень успешные, хорошо отлаженные программы
\textit{PYTHIA} и \textit{POWHEG-BOX}. Они были построены за последние десятилетия наряду с экспериментальными открытиями, и большинство особенностей, видимых в прошлых и настоящих экспериментах, могут быть описаны
ими. Тем не менее, необходимость в более высокой точности для решения задач новых энергетических масштабов происходящих
на \textit{LHC}, сложность конечных состояний в этих масштабах, необходимость обслуживания и желание легкой возможности реализовать новые физические модели, требовалось переписать эти пакеты на современном языке программирования, обеспечивающем более высокий уровень модульности.

Объектно-ориентированные рамки отвечают последним
требования и в связи с предпочтениями сообщества к \textit{С++}, новое поколение генераторов событий
построено на этом языке программирования~\cite{review-sherpa}. Это привело к улучшению повторных реализаций в форме
программ \textit{PYTHIA 8} и \textit{HERWIG ++} -- преемники версий написаных на \textit{Fortran} упомянутых выше
и к созданию генератора событий \textit{SHERPA}~\cite{review-sherpa}.
В связи с этим в последнее десятилетие стали доступны пакеты для расчетов с опережающим порядком. Яркими примерами являются \textit{MCFM} и \textit{NLOJET ++}. Соответствующие методы реализованы, например, в \textit{MC@NLO}, который основан
на фортранской версии \textit{HERWIG}, в \textit{HERWIG ++} и в некоторых более специализированных программах~\cite{review-sherpa}.
Тем не менее, полные расчеты следующего за ведущим порядком, лежащие в основе этих новых методов, очень сложны, и до сегодняшнего дня контролируются только процессы с пятью внешними ветвями. Но
с другой стороны, многие важные экспериментальные сигнатуры зависят от конечных состояний с более высокой кратностью, что
инициировало существенную деятельность по совершенствованию методов и инструментов с точностью на уровне множества ветвей, так что
теперь доступно несколько пакетов, которые могут вычислять соответствующие сечения и генерировать события
полностью автоматизированным способм. Наиболее яркими примерами являются \textit{ALPGEN}, \textit{CompHEP} / \textit{CalcHEP},
\textit{HELAC-PHEGAS}, \textit{MADGRAPH}, \textit{WHIZARD} и \textit{AMEGIC ++}. В настоящее время только библиотека \textit{AMEGIC ++} реализует данный функционал и встроенная в полноценный генератор событий, а именно в среду \textit{SHERPA}. Чтобы
преобразовать многочастичные события на уровне партонов, которые предоставляются этими инструментами в ведущем порядке, в
события уровня адронов, было разработано несколько алгоритмов, все из которых направлены на сохранение логарифмической
точности партонного потока и дополнения его точным результатом возмущающего ведущего порядка для
учитывания кратности струи.

\textit{SHERPA}~\cite{review-sherpa} является аббревиатурой от «Simulation of High Energy Reactions of Particles». Программа является
полная структура генерации событий, которая была построена с нуля и полностью написана в
современный объектно-ориентированный язык программирования \textit{C++}

Построение SHERPA осуществлялось способом, в значительной степени определяемым следующими тремя парадигмами:

\begin{enumerate}
	\item[--] Модульность. Различные физические аспекты реализованы в почти независимых модулях, опираясь на
	небольшое количество структурных и вспомогательных модулей, таких как, например, запись событий и т. д. Модульность
	позволяет, например, иметь более одного генератора матричных элементов или параллельного потока,
	на выбор пользователя. Центральный модуль, \textit{SHERPA}, управляет взаимодействием
	всех других частей и фактической процедуры генерации.
	\item[--] Снизу-вверх. Физические модули обычно разрабатываются сами по себе, проходят испытания и
	проверяются, прежде чем они будут включены в полную структуру генерации событий. Это в свою очередь
	приводит к довольно гибкой, минимальной структуре, лежащей в основе организации генерации событий.
	\item[--] Разделение интерфейса и реализации. Для того, чтобы облегчить два требования выше,
	\textit{SHERPA} опирается на структуру, в которой (почти независимые) физические модули доступны только
	через специфичные для физики обработчики. Эти обработчики помогают \textit{SHERPA} генерировать событие на разных
	этапах, каждый из которых управляется определенной реализацией обработчика фазы события, такой как
	Сигнальный процесс или \textit{Jet Evolution}. Примером такого взаимодействия фазы события и физики является обработчик. Обработчик \textit{Matrix Element Handler}, позволяющий генерировать события на уровне партона либо
	встроенные жестко запрограммированные матричные элементы или генератор матричных элементов \textit{AMEGIC ++}. Обработчик
	актуален для двух этапов события, генерации сигнального процесса и вследствие многоструйного слияния процедуры эволюции струй.
\end{enumerate}

\textit{AMEGIC++} -- это генератор матричных элементов по умолчанию используемый в SHERPA, основанный на диаграммах Фейнмана, которые переводятся в амплитуды спиральности~\cite{review-sherpa}.\textit{ AMEGIC ++} использует библиотеку интеграции фазового пространства Монте-Карло \textit{PHASIC}. Для оценки начального состояния (обратное рассеяние лазера, начальное излучение) и интегралов. В конечном состоянии используется адаптивный многоканальный метод. По умолчанию данный генератор идет вместе с оптимизационным пакетом \textit{Vegas} для отдельных каналов. Кроме того, поддерживается интеграцию оптимизаторов \textit{RAMBO} и \textit{HAAG}~\cite{review-sherpa}.

Эта общая структура полностью отражает парадигму генерации событий в Монте-Карло,
моделирование в четко определенные, почти независимые фазы. Соответственно, каждый обработчик фазы события абстрактным образом инкапсулирует различные аспекты генерации события для реакций с высокой энергией частиц.
Затем эта абстракция заменяется реальной физикой с использованием обработчиков, которые обеспечивают сбор данных о событиях. Данный пакет при генерации может не учитывать мелкие детали базовой физики и ее реализацию в форме физического модуля~\cite{review-sherpa}.

\begin{table}[h!]
	\begin{flushleft}
		\caption{Сравнительный анализ генераторов событий физики
			высоких энергий}
		\label{tab:table1}
		\begin{tabular}{ | c | p{5cm} | p{4cm} | p{4cm} |}
			\hline
			Название & Стпень сложности реализации модели & Время модуляции 10000 столкновений, с & Размер пакета, Мб \\ \hline
			\textit{PYTHIA} & Легко, так как достаточно указать коэффициенты и массы частиц & \hspace{2cm}5 & \hspace{2cm}5 \\ \hline
			\textit{Powheg-Box} & Сложно, так как требуется более детальная настройка начальных параметров системы & \hspace{2cm}10 & \hspace{2cm}20 \\ \hline
			\textit{Sherpa} & Сложно, так как необходимо устанавливать и настраивать дополнительные модули & \hspace{2cm}7 & \hspace{2cm}25 \\
			\hline
		\end{tabular}
	\end{flushleft}
\end{table}

В соответствие с таблицей сравнения~\ref{tab:table1} было принято решение для разработки имитационной медли использовать генератор событий физики
высоких энергий \textit{PYTHIA}. Объектно-орентированый язык программрования, используемый в пакете \textit{PYTHIA}, и ряд готовых модулей в пакете требующих только настроичных входных параметров способствуют более простой и быстрой реализации имметационной моедли для генерации одного канала событий. В пакетах \textit{Powheg-Box} и \textit{Sherpa} время модуляции десяти тысячи столкновений протонов на несколько секунд выше чем в пакете \textit{PYTHIA}, что сложится в часы при расчете значений для всех ${M}_{{Z}^{\prime}}$.
% и точность
% Генерации одного канала событий, что уменьшает затраты процессорного времени


% https://arxiv.org/format/0811.4622