Для реализации поставленной задачи исследования в магистерской
диссертации был выбран пакет моделирования процессов столкновения элементарных частиц при высоких энергиях на ускорителях элементарных частиц \textit{PYTHIA} на языке программирования \textit{С++}, а также принято решение о реализации всего программного
комплекса на языке программирования \textit{Java}.
В качестве веб интерфейса был использован фреймворк \textit{Angular} на языке программирования \textit{JavaScript}. Так же была использована и бибилотека \textit{d3js} для отрисовки графиков в виде \textit{SVG} изображений и взаимодейсвия пользователся с интерфейсом приложения.

Когда говорят о научных основах проектирования пользовательских
интерфейсов, в первую очередь упоминают термин \textit{Human-Computer Interaction}
(\textit{HCI}) – «взаимодействие человека и компьютера». В странах Запада \textit{HCI} это
является целой профессией, ей обучают в университетах, издается много
журналов по этой теме, существует большое количество \textit{web}-сайтов~\cite{user-interface}.
Составными частями \textit{HCI} являются:

\begin{enumerate}
	\item[--] человек (пользователь);
	\item[--] компьютер;
	\item[--] их взаимодействие.
\end{enumerate}

Пользовательский интерфейс \textit{user interface} (\textit{UI}) – является своеобразным
коммуникационным каналом, по которому осуществляется взаимодействие
пользователя и компьютера.

Лучший пользовательский интерфейс – это такой интерфейс, которому
пользователь не должен уделять много внимания, почти не замечать его. В руках
пользователя интерфейс пользователя должен служить инструментом для
достижения цели. Такой интерфейс называют прозрачным – пользователь
смотрит сквозь него на свою работу.

Чтобы создать эффективный интерфейс, который делал бы работу с
программным комплексом эффективной, нужно понимать, какие задачи будут
решать пользователи с помощью данной программного комплекса и какие
требования к интерфейсу могут возникнуть у пользователей. Большую роль в
разработке интерфейса играет интуиция – если разработчик сам терпеть не
может некрасивые и неудобные интерфейсы, то при создании собственного
программного комплекса он будет чувствовать, где и какой именно элемент
нужно убрать или добавить. Необходимо иметь художественный вкус, чтобы
понимать, что именно придаст интерфейсу красоту и привлекательность.

Западные исследователи в области \textit{HCI} сформулировали основные
принципы проектирования пользовательских интерфейсов компьютерных
программ~\cite{user-interface}. Как и в любой другой отрасли ИТ, существует довольно много
различных методик и классификаций. Можно сформировать три положения
говоря об общих принципах проектирования пользовательского интерфейса:

\begin{enumerate}
	\item[--] программный комплекс должен помогать выполнить задачу, а не
	становиться этой задачей;
	\item[--] при работе с программой пользователь не должен думать, что он не
	понимает программу;

	\item[--] программный комплекс должен работать так, чтобы пользователь не
	считал компьютер бесполезным инструментом.
\end{enumerate}

Конечно, глубина проработки интерфейса и степень его адаптивности под
нужды пользователя в программных комплексах в основном зависит от усилий
их авторов, а не от характеристик аппаратного обеспечения. Однако у
большинства пользователей компьютер ассоциируется именно с программными
комплексами, которые на нем работают, и плохое впечатление от использования
программного обеспечения автоматически переносится на сам компьютер.

Открыв начальную веб-страницу приложения в любом из доступных браузеров пользователь увидит сообщение с описанием проекта, как показано на рисунке~\ref{fig:welcome-page}.

\begin{figure}[!h]
	\centering
	\includegraphics[width=\textwidth]{figures/welcome-page.png}
	\caption{Начальная \textit{web}-страница}
	\label{fig:welcome-page}
\end{figure}

Сообщество разработчиков фреймворка \textit{Angular} разрабатывает дополнительные компоненты \textit{Angular Material Design} и предлагает использовать их для быстой разработки приложения с нуля. 

\textit{Material Design} -- визуальный язык, представлен в 2014 году \textit{Google}, используется чаще всего в мобильных приложения. Пример использования \textit{Material Design} можно увидеть во многих мобильных приложения \textit{Google} (\textit{Play, Music, Books} и т.д.), а также в \textit{Chrome OS}. \textit{Material Design} упрощает разработчикам настройку \textit{UI}, сохраняя при этом удобный интерфейс приложений. \textit{Angular Material} состоит из набора предустановленных компонентов \textit{Angular}. \textit{Anglate Material} стремится обеспечить расширенный и последовательный пользовательский интерфейс. В то же время он дает возможность контролировать, как ведут себя разные компоненты.

Для каждого пользователя на начально странице приложения распологается навигационное меню, котрое показано на рисунке~\ref{fig:menu}. Так как разработанное приложение является одностраничным то переход по пунктам меню не перезагружает страницу полностью, а лишь догружает необходимые компоненты.

\begin{figure}[!h]
	\centering
	\includegraphics[width=0.5\textwidth]{figures/menu.png}
	\caption{Главное меню приложения}
	\label{fig:menu}
\end{figure}
\vspace{5cm}
Из главного меню доступен переход на следующие страницы приложения: 

\begin{enumerate}
	\item[--] <<Введение>> -- начальная страница \textit{web}-приложения;
	\item[--] <<График>> -- страницы с основным графиком и панелью для ввода параметров и отравки запроса на вычисление;
	\item[--] <<Результат>> -- страница предосталяющая полученные результаты на угол смешивания ${Z}^{\prime}$-бозонов в модели \textit{SSM};
	\item[--] <<Статистика>> -- страница со статистикой приложения;
\end{enumerate}

Перейдя на страницу <<График>> пользователь увидет пустой график распределения теоретического сечения кросс-секции $\sigma \times Br({Z}^{\prime} \rightarrow {W}^{+}{W}^{-})$ для ${Z}^{\prime}_{SSM}$ и множество панелей управления, которые позволяют отправить запрос на начало эмуляции процесса $pp \rightarrow {W}^{+}{W}^{-} + X$ в протон-протонном столкновении. 

Для начала старта генерации собыйтий необходимо заполнить значение кси ($\xi$), количество моделируемых событий и количество циклов моделирования одного значения на графике для одного значения массы ${M}_{{Z}^{\prime}}$. Данная панель показана на рисуноке~\ref{fig:request-line}.
% ... с началом массы ${M}_{{Z}^{\prime}}$ и шагом


\vspace{16pt}
\begin{figure}[!h]
	\centering
	\includegraphics[width=\textwidth]{figures/request-line.png}
	\caption{Панель старта расчета линии значений}
	\label{fig:request-line}
\end{figure}

После выставления нужных пользователю параметров, необходимо нажать кнопку <<Отправить>> после чего данные отправяться на сервер и результат будет перерисован на интерфейсе приложения. Результаты вычислений будут добавлены на основной график в реальном времене, поэтому пользователю не нужно обновлять страницу с графиком.

Как только запрос был отправлен на сервер соездается \textit{websocket} соединение между \textit{web}-браузером пользователя и серверным приложением, тем самым подписывая клинта на получени денных от сервера, как только они будут готовы.

На графике кросс-секции $\sigma \times Br({Z}^{\prime} \rightarrow {W}^{+}{W}^{-})$ для ${Z}^{\prime}_{SSM}$ отриросовывается линия с новыми значениями в то время, как серверное приложение начинает параллельно для нескольких значений массы ${M}_{{Z}^{\prime}}$ моделировать столкновение протонов.

Так как в реальном столкновении протонов рождается очень большое количество различных частиц то моделирование такой задачи при помощи генератора \textit{PYTHIA} требудет значительных ресурсов центрального процессора сервера и занимает продолжительное время. Приложение созданное в рамках данной работы имеет \textit{web}-интерфейс, что позволяет сразу нескольком пользователям работать в нем и отправлять запросы на вычичления. В связи с этим вводится ограничение на количество параллельно зарпущеных процессов моделирования для каждого пользователя.

Даже самая простая задача расчета модели занимает большое время. К примеру запрос пользователя расчитать линию для всех значений ${M}_{{Z}^{\prime}}$, а в рамках данного проекта это пять тысячь точек на графике от 0 до 5000 ГэВ, и количеством генерируемых собыйтий займет 9,7 часа. 

В таком простом случае полученное значение будет подвержено значительной статистической ошибки. Чтобы убрать данную статистическую ошибку тербуются дополнительные циклы имметационного моделирования, что увеличивает затраченное время в разы. 

Под панелью для запроса вычислений распологается список ранее отправильных запросов, как показано на рисунке~\ref{fig:preferences}. Нажав на крестик <<X>> можно удалить линию на графике.

Все запросы на вычисление сохраняются для конкретного пользователя и загружаются из локального хранилища данных при обновлении страницы.

\begin{figure}[!h]
	\centering
	\includegraphics[width=\textwidth]{figures/preferences.png}
	\caption{Список отображаемых линий}
	\label{fig:preferences}
\end{figure}

Так как вычисления могут занять довольно много времени то вычисляются не все значения массы ${M}_{{Z}^{\prime}}$. Устанавливается шаг для вычислений к примеру 100 ГэВ и в качестве входного значения массы подставляются значения в 100 ГэВ, 200 ГэВ и т.д. Этот подход экономит значительное коливество ресурсов вычислительноый машины и экономит время для полусения общей картины поведения распределения ограничений на угол смешивания. Дополнительно на стороне сервера предусмотрено хранение уже рассчитаных результатов по выходным данным имметационного моделирования.

Как видно из рисунка~\ref{fig:offline-calc} у пользователя есть возможно отправить <<Отложенный>> запрос на вычисление для удобства вычисления линий. Отложенный запрос означает, что запрос добавиться в очередь на вычислесения на стороне сервера. Запрос будет сохраняться до того момента пока вычислительные ресурсы не осободяться и только после этого запрос обработается и начнется процесс имметационного моделирования.

\begin{figure}[!h]
	\centering
	\includegraphics[width=0.4\textwidth]{figures/offline-calc.png}
	\caption{Переключатель в <<отложенный>> режим отправки запросов}
	\label{fig:offline-calc}
\end{figure}

Предусмотрена возможность отправлять запросы на вычисление только одной точки на графике. Данная возможно позволяет узнать точное значение кросс-секции $\sigma \times Br({Z}^{\prime} \rightarrow {W}^{+}{W}^{-})$ для конкретной массы ${Z}^{\prime}_{SSM}$-бозона и не тратить время на вычисления значений всех масс ${M}_{{Z}^{\prime}}$ для заданной $\xi$. Панель позволяющая отправлять такие запросы представлена на картинке~\ref{fig:request-point}.

\vspace{16pt}
\begin{figure}[!h]
	\centering
	\includegraphics[width=\textwidth]{figures/request-point.png}
	\caption{Панель старта расчета одного значения}
	\label{fig:request-point}
\end{figure}

Общий процесс вычиления всех значений для масс ${M}_{{Z}^{\prime}}$ отображается в виде прогресс бара в самом низу веб страницы. Как показано на рисунке~\ref{fig:progress} данный элемент не имеет каких-либо точек взаимодействия с пользователем и несет лишь информативный характер.

\begin{figure}[!h]
	\centering
	\includegraphics[width=\textwidth]{figures/progress.png}
	\caption{Прогресс вычислений}
	\label{fig:progress}
\end{figure}

Компонент <<Прогресс вычислений>> содержит комплексную информацию о ходе вычислений: количество вычисляемых значений в данный момент времени, лимит вычисляемых точек в одно время для сессии одного пользователя, очередь запросов от пользователя и количество успешно полученных результатов.

Так как вычисления могут занять довольно много времени то вычисляются только текущие запросы для текущей сессии пользователя. К примеру если, лимит на вычисления равен десять запросов в одно время и пользователь запустил процесс вычиления целой линии значений, а после закрыл старницу \textit{web}-приложения то будут произведены вычисления только этих десяти значений. Далнейшие вычисления будут возобновлены, когда пользователь внонь откроет страницу \textit{web}-интерфейса или другой пользователь отправить запрос на вычисление с теми же входными параметрами.

Интерфейс результатов и вычислений выполнен в виде \textit{SVG} элемента и состоит из графика кросс-секции $\sigma \times Br({Z}^{\prime} \rightarrow {W}^{+}{W}^{-})$ для масс ${M}_{{Z}^{\prime}}$ (рисунок~\ref{fig:graph-1}).

\begin{figure}[!h]
	\centering
	\includegraphics[width=\textwidth]{figures/graph-1.png}
	\caption{Панель старта расчета линии значений}
	\label{fig:graph-1}
\end{figure}

На графике отображены статические элементы полученные из математических формул и вычислений, а также данные не вычисляемые в рамках данной диссертации и полученные от коллаборации \textit{ATLAS}, такие  как \textit{reference model}, \textit{observed} значения, \textit{expected} значения с областью в два стандартных отклонения~\cite{2part-pankov}.

Представленный на рисунке~\ref{fig:graph-1} график имеет элемент управления - вертикальная красная линия. Данная линия добавлена, чтобы помочь пользователю определить занчание кросс-секции для массы ${M}_{{Z}^{\prime}}$ и выбранным углом смешивания $\xi$. Близлежащие значения к данной вертикальной линии будут подсвечены на графике и их точное значение всегда отображается в правой части графика.

Перейдя на вкладку <<Результат>> из меню приложения (рисунок~\ref{fig:menu}) пользователю будет предоставлен результат работы приложения, который показан на рисунке~\ref{fig:graph-result}.

Результат представляет собой график зависемости угла смешивания $\xi$ и массы ${M}_{{Z}^{\prime}}$ и отражает ограничения на угол смешивания ${Z}^{\prime}$-бозонов в модели \textit{SSM}. 

Значениями ломаной \textit{Observed} являются точки ($\xi$) пересечения моделируемых линий и кривой \textit{expected}, которые отображены на рисунке~\ref{fig:graph-1}. В данной диссертационной работе при помощи разработанной имметационной модели рождения $Z^\prime$ - бозонов в протон-протонных столкновениях с учетом эффектов $Z$ - $Z^\prime$ смешивания были получены ограничение на угол смешивания ${Z}^{\prime}$-бозонов для до светимости 3000 фб${}^{−1}$ и равняется $6\times{10}^{-5}$.

\begin{figure}[!h]
	\centering
	\includegraphics[width=\textwidth]{figures/graph-result.png}
	\caption{Ограничения на угол смешивания $Z$-${Z}^{\prime}$ полученные из обработки данных имитационного моделирования}
	\label{fig:graph-result}
\end{figure}

Изходя из того, что большая часть использованной литературы при исследовании проблемы описанной в данной диссертационной работе является англоязычной было принято решение реализовать возможность смены языка на котором отображается текст приложения. Такой функционал добавлен в правом верхнем углу \textit{web}-интерфейса, рисунок~\ref{fig:language-switch}.

\vspace{18pt}
\begin{figure}[!h]
	\centering
	\includegraphics[width=0.3\textwidth]{figures/language-switch.png}
	\caption{Смена языка \textit{web}-страницы}
	\label{fig:language-switch}
\end{figure}

