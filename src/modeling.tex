Для выполнения анализа процесса функционирования технической
системы при случайных внешних воздействиях возникает необходимость
моделирования этих воздействий. Реализации функции внешних воздействий на
ЭВМ представляются в виде случайных последовательностей (значения
воздействий в дискретные моменты времени), отображающих дискретные
случайные процессы с заданными вероятностными характеристиками~\cite{modeling:2004}.

При моделировании стационарного случайного воздействия с нормальным
распределением достаточно сформировать случайную последовательность с
заданной корреляционной функцией. В основу алгоритмов формирования таких
процессов положено линейное преобразование стационарной
последовательности ${x}_{k}^{N}$ независимых случайных чисел, имеющих нормальное
распределение, в последовательность ${q}_{k}$. При этом случайная
последовательность ${x}_{k}^{N}$ подается на вход дискретного линейного фильтра,
формирующего на выходе дискретный случайных процесс с заданной
корреляционной функцией.

Алгоритмы формирования дискретных случайных процессов задаются
рекуррентными соотношениями: