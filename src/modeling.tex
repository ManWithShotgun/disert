Для выполнения анализа процесса функционирования технической
системы при случайных внешних воздействиях возникает необходимость
моделирования этих воздействий. Реализации функции внешних воздействий на
ЭВМ представляются в виде случайных последовательностей (значения
воздействий в дискретные моменты времени), отображающих дискретные
случайные процессы с заданными вероятностными характеристиками~\cite{modeling:2004}.

При моделировании стационарного случайного воздействия с нормальным
распределением достаточно сформировать случайную последовательность с
заданной корреляционной функцией. В основу алгоритмов формирования таких
процессов положено линейное преобразование стационарной
последовательности ${x}_{k}^{N}$ независимых случайных чисел, имеющих нормальное
распределение, в последовательность ${q}_{k}$. При этом случайная
последовательность ${x}_{k}^{N}$ подается на вход дискретного линейного фильтра,
формирующего на выходе дискретный случайных процесс с заданной
корреляционной функцией.

Алгоритмы формирования дискретных случайных процессов задаются
рекуррентными соотношениями:
\begin{equation} \label{eq:model1}
{q}_{k} = \sum_{i=0}^{N} c_i x_{k-i}, k=0,1,2,...
\end{equation}

При математическом моделировании процессов функционирования
технической системы для определения вероятностных характеристик
используют реализации дискретных случайных процессов, получаемых в
результате вычислительного эксперимента на ЭВМ. Дискретные случайные
процессы характеризуют изменение во времени фазовых координат и выходных
параметров. технической системы в условиях случайных воздействий внешней
среды. Значения фазовых координат получают в процессе интегрирования
системы дифференциальных уравнений математической модели, а значения
выходных параметров вычисляют на основе функциональных зависимостей
между ними и фазовыми координатами. Задачей анализа процесса
функционирования технической системы в этом случае является получение
статистических оценок вероятностных характеристик фазовых координат и
выходных параметров, характеризующих качество и эффективность системы, и
оценка степени выполнения технических требований на эти параметры~\cite{modeling:2004}.
