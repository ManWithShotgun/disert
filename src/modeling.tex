Для выполнения анализа процесса функционирования технической
системы при случайных внешних воздействиях возникает необходимость
моделирования этих воздействий. Реализации функции внешних воздействий на
ЭВМ представляются в виде случайных последовательностей (значения
воздействий в дискретные моменты времени), отображающих дискретные
случайные процессы с заданными вероятностными характеристиками~\cite{modeling:2004}.

При моделировании стационарного случайного воздействия с нормальным
распределением достаточно сформировать случайную последовательность с
заданной корреляционной функцией. В основу алгоритмов формирования таких
процессов положено линейное преобразование стационарной
последовательности ${x}_{k}^{N}$ независимых случайных чисел, имеющих нормальное
распределение, в последовательность ${q}_{k}$. При этом случайная
последовательность ${x}_{k}^{N}$ подается на вход дискретного линейного фильтра,
формирующего на выходе дискретный случайных процесс с заданной
корреляционной функцией.

Алгоритмы формирования дискретных случайных процессов задаются
рекуррентными соотношениями:
\begin{equation} \label{eq:model1}
{q}_{k} = \sum_{i=0}^{N} c_i x_{k-i}, k=0,1,2,...
\end{equation}

\begin{equation} \label{eq:model2}
{q}_{k} = \sum_{i=0}^{L} a_l x_{k-i} - \sum_{j=0}^{L} b_j q_{k-j}, k=0,1,2,...
\end{equation}
где $a_l$, $b_j$, $c_i$ – параметры алгоритмов, определяемые по корреляционной функции ${R}_{q}(\tau)$, формируемого дискретного случайного процесса $q_k$.

Начальные значения $q_k$ при $k = 0$ в этих алгоритмах для простоты можно
выбирать нулевыми. При этом начальный участок моделируемого процесса
будет несколько искажен переходным процессом, по окончании значения $k$
становится стационарной.

Для получения коэффициентов $a_l$, $b_j$, $c_i$, входящих в выражения
скользящего суммирования (\ref{eq:model1}) и (\ref{eq:model2}), применяются: разложения на ряд Фурье
функции спектральной плотности; решение системы нелинейных
алгебраических уравнений, правая часть которой определяется исходной
корреляционной функцией; метод факторизации и другие.

Рассмотрим наиболее часто встречающиеся на практике корреляционные
функции ${R}_{q}(\tau)$ и алгоритмы моделирования случайных процессов, основанные на
преобразовании последовательности ${x}_{k}^{N}$ независимых нормально
распределенных чисел с математическим ожиданием $m_x = 0$ и дисперсией
${\sigma}_{x}^{2} = 1$ в
последовательность $q_k$ , характеризующего корреляционной функцией.


\begin{equation} \label{eq:model3}
{R}_{q}(\tau) = {R}_{q}(kh) = M[q_l q_{l+k}], k=0,1,2,...
\end{equation}
где $h$ – шаг дискретизации независимой переменной $t$.

При статистическом анализе случайных процессов в технических системах
описание характеристик внешних воздействий обычно дается в виде
корреляционных функций ${R}_{q}(\tau)$ или спектральных плотностей $G_q(\omega)$. Эти
функции получают путем статистической обработки результатов эксперементов
графики корреляционных функций аппроксимируют некоторыми функциями~\cite{modeling:2004}. Наиболее часто используют экспоненциальное и экспненциальнокоссинусные функции:


\begin{equation} \label{eq:model4}
{R}_{q}(\tau) = {\sigma}_{x}^{2} e^{-\alpha \tau}
\end{equation}

\begin{equation} \label{eq:model5}
{R}_{q}(\tau) = {\sigma}_{x}^{2} e^{-\alpha \tau} \cos\beta\tau, k=0,1,2,...
\end{equation}
где ${\sigma}_{x}^{2}$ -- дисперсия возмущающего воздействия q(t);\\
$\alpha$ -- коэффициент, характеризующий затухание корреляционной функции;\\
$\beta$ -- коэффициент, характеризующий колебательный процесс.


Разделив корреляционную функцию ${R}_{q}(\tau)$ на ${\sigma}_{x}^{2}$, получим нормированную
корреляционную функцию ${\rho}_{q}(\tau)$. Корреляционная функция, определяемая
выражением \ref{eq:model5}, относится к случайному процессу, содержащему
периодическую составляющую.

Спектральная плотность стационарного случайного процесса $G_q(\omega)$
представляет собой функцию круговой частоты $\omega$, которая равна
преобразованию Фурье ковариационной функции $K(\tau)$ этого процесса:

\begin{equation} \label{eq:model6}
G_q(\omega) = \int_{-\infty}^{\infty} K(\tau) e^{-j\omega\tau} d\tau
\end{equation}

После преобразования выражения (\ref{eq:model6}) для центрированного случайного
пресса получим известное соотношения для области положительных частот:

\begin{equation} \label{eq:model7}
G_q(\omega) = \frac{2}{\pi}\int_{-\infty}^{\infty} K(\tau) \cos\omega\tau d\tau
\end{equation}

Для корреляционной функции ${R}_{q}(\tau)$ справедливо выражение:

\begin{equation} \label{eq:model8}
R_q(\tau) = \int_{0}^{\infty} G_q(\omega) \cos\omega\tau d\omega
\end{equation}

Поскольку при $\tau = 0$ $R_q(0) = {\sigma}_{x}^{2}$,то на основании выражения (\ref{eq:model8}) получаем:

\begin{equation} \label{eq:model9}
{\sigma}_{q}^{2} = \int_{0}^{\infty} G_q(\omega) d\omega
\end{equation}

Следовательно, площадь, ограниченная графиком функции $G_q(\omega)$ и осью
частот $\omega$, представляет собой дисперсию стационарного случайного процесса.

Перейдем к описанию алгоритмов формирования реализаций дискретного
случайно процесса с корреляционными функциями \ref{eq:model4} и \ref{eq:model5}.

Последовательность ординат случайного процесса с корреляционной
функцией \ref{eq:model4} получают по формуле:

\begin{equation} \label{eq:model10}
{q}_{k} = a_0 x_{k}^{N} + b_1 q_{k-1}, k=0,1,2,...
\end{equation}
где $a_0 = \sigma_q \sqrt{1 - b_1^2}$;\\
$b_1 = e^{-\alpha h}$;\\
$h$ -- шаг дискретизации независимой переменной $t$;\\
$x_k^N$ -- значения нормально распределенной случайной величины $X^N$ с
параметрами $m_x = 0$ и $\sigma_x = 1$.

Последовательность ординат случайного процесса с корреляционной
функцией (\ref{eq:model5}) получают по формуле:

\begin{equation} \label{eq:model11}
{q}_{k} = a_0 x_{k}^{N} + a_1 x_{k-1}^{N} +  b_1 q_{k-1} +  b_2 q_{k-2}, k=0,1,2,...
\end{equation}
где $a_0 = \sigma_q b_0$;\\
$a_1 = \sigma_q a_0 / b_0$;\\
$b_0 = \sqrt{(c_1 + \sqrt{c_1^2 - 4c_0^2})/2}$;\\
$b_1 = 2e^{-\alpha h} \cos\beta h$;\\
$b_2 = -e^{-2\alpha h}$;\\
$c_0 = e^{-\alpha h} (e^{-2\alpha h} - 1) \cos\beta h$;\\
$c_1 = 1 - e^{-4\alpha h}$.

Если корреляционная функция случайного процесса представляет собой
сумму выражений~\ref{eq:model4} и ~\ref{eq:model5}, то значение $q_k$ равно сумме ординат, вычисленных
по формулам~\ref{eq:model10} и~\ref{eq:model11}. При этом в этих формулах должны быть
независимыми последовательностями нормально распределенных величин с
параметрами $m_x = 0$ и $\sigma_x = 1$.

При математическом моделировании процессов функционирования
технической системы для определения вероятностных характеристик
используют реализации дискретных случайных процессов, получаемых в
результате вычислительного эксперимента на ЭВМ. Дискретные случайные
процессы характеризуют изменение во времени фазовых координат и выходных
параметров. технической системы в условиях случайных воздействий внешней
среды. Значения фазовых координат получают в процессе интегрирования
системы дифференциальных уравнений математической модели, а значения
выходных параметров вычисляют на основе функциональных зависимостей
между ними и фазовыми координатами. Задачей анализа процесса
функционирования технической системы в этом случае является получение
статистических оценок вероятностных характеристик фазовых координат и
выходных параметров, характеризующих качество и эффективность системы, и
оценка степени выполнения технических требований на эти параметры~\cite{modeling:2004}.
