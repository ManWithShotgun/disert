В результате проведённого сравнительного анализа существующих Монте-Карло генераторов для моделирования процессов взаимодействия элементарных частиц выявлено, что в них отсутствует реализация моделей ${Z}^{\prime}$-бозонов с учётом эффектов $Z-{Z}^{\prime}$ смешивания. 

На сегодняшний день, наиболее универсальным и широко используемым в физике частиц генератором является \textit{PYTHIA}. Поэтому в настоящей работе основной фокус направлен на расширение возможностей генератора \textit{PYTHIA} путём включения в него поддержки моделирования эффектов ${Z}^{\prime}$-бозонов с учётом $Z-{Z}^{\prime}$ смешивания.

Таким образом, можно выделить следующие актуальные задачи:
\begin{enumerate}
	\item[--] разработка имитационной модели процесса рождения ${Z}^{\prime}$-бозонов в протон-протонных столкновениях с учетом эффектов $Z$-${Z}^{\prime}$ смешивания;
	
	\item[--] проектирование и разработка программного модуля для имитационного моделирования процесса
	рождения ${Z}^{\prime}$-бозонов в протон-протонных столкновениях с учетом эффектов $Z$-${Z}^{\prime}$ смешивания в условиях эксперимента \textit{ATLAS} на Большом адронном коллайдере;
	
	\item[--] оценка ограничений на углы смешивания ${Z}^{\prime}$-бозонов в процессе рождения ${W}^{+}$${W}^{-}$ пар в протон-протонных столкновениях
	в условиях экспериментов на Большом адронном коллайдере, рассчитаные для интегральной светимости 1000 фб${}^{−1}$ и 3000 фб${}^{−1}$.
	
\end{enumerate}