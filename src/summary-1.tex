Основной особенностью всех пакетов является то, что они разработаны в виде библиотек для моделирования процессов столкновения частиц при высоких энергиях осуществляющие генерацию методом Монте-Карло физических событий, однако не было выявлено системы, пригодной для нахождения оценки ограничений на углы смешивания ${Z}^{\prime}$-бозонов в процессе рождения ${W}^{+}$${W}^{-}$ пар в протон-протнных столкновениях в условиях экспериментов на Большом адронном коллайдерев.

Таким образом, можно выделить следующие актуальные задачи:
\begin{enumerate}
	\item[--] разработка математической модели процесса рождения ${Z}^{\prime}$-бозонов в протон-протнных столкновениях с учетом эффектов $Z$-${Z}^{\prime}$ смешивания;
	
	\item[--] проектирование и разработка программного модуля для имитационного моделирования процесса
	рождения ${Z}^{\prime}$-бозонов в протон-протнных столкновениях с учетом эффектов $Z$-${Z}^{\prime}$ смешивания в условиях эксперимента \textit{ATLAS} на Большом адронном коллайдере;
	
	\item[--] проведение оценки ограничений на углы смешивания ${Z}^{\prime}$-бозонов в процессе рождения ${W}^{+}$${W}^{-}$ пар в протон-протнных столкновениях
	в условиях экспериментов на Большом адронном коллайдере, рассчитаные для интергальной светимости 1000 фб${}^{−1}$ и 3000 фб${}^{−1}$.
	
\end{enumerate}