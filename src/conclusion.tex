В результате проделанной работы была предложена математическая
модель процесса
рождения ${Z}^{\prime}$-бозонов в протон-протнных столкновениях с учетом эффектов $Z$-${Z}^{\prime}$ смешивания в условиях эксперимента \textit{ATLAS} на Большом адронном коллайдере.

По математической модели была разработана имитационная модель процесса рождения ${Z}^{\prime}$-бозонов в протон-протнных столкновениях с последующим распадом на пару ${W}^{+}{W}^{-}$ бозонов, отличающиюся от известных моделей учетом эффектов $Z$-${Z}^{\prime}$ смешивания.

Выполнена обработка экспериментальных данных коллабораци  \textit{ATLAS} на Большом адронном коллайдере \textit{LHC} (с энергией 13 ТэВ и светимостью 36,1 фб${}^{−1}$) по измерению процесса рождения ${W}^{+}$${W}^{-}$ пар в протон-протонных столкновениях.

Для удобного использования имитационной модели было разработано необходимое программное обеспечение для имитационного моделирования процесса
рождения ${Z}^{\prime}$-бозонов в протон-протнных столкновениях с учетом эффектов $Z$-${Z}^{\prime}$ смешивания в условиях эксперимента \textit{ATLAS} на Большом адронном коллайдере, которое отличается от существующих тем, что программный модуль собран в \textit{Docker} образ позволяющий быстро проводить имитационное моделирование без установки необходимых программных средств

Было проведено имметационное моделирование и по результатам моделирования были получены ограничения на угол смешивания ${Z}^{\prime}$-бозонов в модели \textit{SSM}, которые составили $\xi$ < 0,0004, что на порядок лучше результатов полученных ранее из глобального анализа электрослабых данных. А так же рассчитаны ограничения для светимостей 1000 фб${}^{−1}$ и 3000 фб${}^{−1}$, которые составили ${10}^{-4}$ и $6\times{10}^{-5}$, соотвественно.

Была проведена верификация полученных результатов с
работой~\cite{2part-pankov}. Все полученные результаты успешно верифицировались и было найдено, что разработанная имитационная модель в программном комплексе имеет точность 90\%.

Разработанный программный комплекс может быть использован для дальнейшего узучения процесса рождения ${Z}^{\prime}$-бозонов в протон-протнных столкновениях с учетом эффектов $Z$-${Z}^{\prime}$ смешивания в условиях эксперимента \textit{ATLAS} на Большом адронном коллайдере, накапливая результаты имметационного моделирования тем самым оценивая ограничения на угол смешивания ${Z}^{\prime}$-бозонов для больших светимостей. 