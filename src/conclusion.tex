В результате проделанной работы разработана имитационная модель процесса рождения ${Z}^{\prime}$-бозонов в протон-протонных столкновениях с последующим распадом на пару ${W}^{+}{W}^{-}$ бозонов, отличающиеся от известных моделей учетом эффектов $Z$-${Z}^{\prime}$ смешивания.

Выполнена обработка экспериментальных данных коллабораци  \textit{ATLAS} на Большом адронном коллайдере \textit{LHC} (с энергией 13 ТэВ и светимостью 36,1 фб${}^{−1}$) по измерению процесса рождения ${W}^{+}$${W}^{-}$ пар в протон-протонных столкновениях.
Для удобного использования имитационной модели было разработано необходимое программное обеспечение для имитационного моделирования процесса
рождения ${Z}^{\prime}$-бозонов в протон-протонных столкновениях с учетом эффектов $Z$-${Z}^{\prime}$ смешивания в условиях эксперимента \textit{ATLAS} на Большом адронном коллайдере, которое отличается от существующих тем, что программный модуль собран в \textit{Docker} образ позволяющий быстро проводить имитационное моделирование без установки необходимых программных средств.

Выполнено имитационное моделирование и по результатам моделирования получены ограничения на угол смешивания ${Z}^{\prime}$-бозонов в модели \textit{SSM}, которые составили $\xi$ < 0,0004, что на порядок лучше результатов полученных ранее из глобального анализа электрослабых данных. А так же впервые рассчитаны ограничения для значений светимости 1000 фб${}^{−1}$ и 3000 фб${}^{−1}$, которые составили ${10}^{-4}$ и $6\times{10}^{-5}$, соотвественно.

Проведена верификация полученных результатов с
работой~\cite{2part-pankov}. Все полученные результаты успешно верифицированы. Уровень достоверности полученных результатов составляет 95\%.

Разработанный программный комплекс может быть использован для дальнейшего изучения процесса рождения ${Z}^{\prime}$-бозонов в протон-протонных столкновениях с учетом эффектов $Z$-${Z}^{\prime}$ смешивания в условиях эксперимента \textit{ATLAS} на Большом адронном коллайдере. Эксперементальная программа Большого адронного коллайдера на  ближайшие годы предполагает увеличение энергии сталкивающихся протонов до 14 ТэВ и увеличения значений светимости до 1000 фб${}^{−1}$, а возможно и до 3000 фб${}^{−1}$. Разработанный программный комплекс может быть использован для обработки новых эксперементальных данных для получения ограничений на угол смешивания ${Z}^{\prime}$-бозонов и их массу.