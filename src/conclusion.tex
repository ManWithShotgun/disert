Выполнена обработка экспериментальных данных коллабораци  \textit{ATLAS} на Большом адронном коллайдере \textit{LHC} (с энергией 13 ТэВ и светимостью 36,1 фб${}^{−1}$) по измерению процесса рождения ${W}^{+}$${W}^{-}$ пар в протон-протонных столкновениях.

Была разработана математическая модель и имментационная модель, проведено имметационное моделирование и по результатам моделирования были получены ограничения на угол смешивания ${Z}^{\prime}$-бозонов в модели \textit{SSM}, которые составили $\xi$ < 0,0004, что на порядок лучше результатов полученных ранее из глобального анализа электрослабых данных. А так же рассчитаны ограничения для светимостей 1000 фб${}^{−1}$ и 3000 фб${}^{−1}$, которые составили ${10}^{-4}$ и $6\times{10}^{-5}$, соотвественно.